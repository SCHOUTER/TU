\documentclass[11pt,a4paper]{article}

% ---------------------------------------------------------
% Encoding, Fonts, Typography
% ---------------------------------------------------------
\usepackage[utf8]{inputenc}
\usepackage[T1]{fontenc}
\usepackage{lmodern}
\usepackage{microtype}

% ---------------------------------------------------------
% Language
% ---------------------------------------------------------
\usepackage[ngerman,english]{babel}

% ---------------------------------------------------------
% Math
% ---------------------------------------------------------
\usepackage{amsmath, amssymb, amsfonts}
\usepackage{mathtools}

% ---------------------------------------------------------
% Tables
% ---------------------------------------------------------
\usepackage{booktabs}
\usepackage{tabularx}

% ---------------------------------------------------------
% Figures & Graphics
% ---------------------------------------------------------
\usepackage{graphicx}
\usepackage{tikz}

% ---------------------------------------------------------
% Lists
% ---------------------------------------------------------
\usepackage{enumitem}
\setlist{nosep}

% ---------------------------------------------------------
% Units & Numbers
% ---------------------------------------------------------
\usepackage{siunitx}

% ---------------------------------------------------------
% Hyperlinks
% ---------------------------------------------------------
\usepackage[hidelinks]{hyperref}
\usepackage{xurl}

% ---------------------------------------------------------
% Global layout improvements
% ---------------------------------------------------------
\clubpenalty=10000
\widowpenalty=10000
\emergencystretch=3em

% ---------------------------------------------------------
% Title
% ---------------------------------------------------------
\title{\"Ubung 8: Gruppe 28}
\author{
    Niclas Kusenbach, 360227 \and
    Alicia Bayerl, 2633336 \and
    Mohamed Naceur Hedhili, 2957151 \and
    Selma Naz Öner, 2662640
}

\begin{document}
\maketitle
\section*{Aufgabe 8.1: Transformationen (Quelle: Folien 13, 21, 35-44)}

\subsection*{a) Zuordnung der Begriffe (Visuelle Analyse)}
Basierend auf der grafischen Darstellung der Teekannen im Koordinatensystem:

\begin{itemize}
   \item \textbf{(a) Rotation}
   \item \textbf{(b) Translation}
   \item \textbf{(c) Scherung}
   \item \textbf{(d) Skalierung}
\end{itemize}

\subsection*{b) Zuordnung der Matrizen}

\begin{itemize}
   \item \textbf{(a) Skalierung} [Quelle: Folie 35]
         \[
            \begin{pmatrix}
               5 & 0 & 0 & 0 \\
               0 & 5 & 0 & 0 \\
               0 & 0 & 5 & 0 \\
               0 & 0 & 0 & 1
            \end{pmatrix}
         \]
         Dies ist eine Diagonalmatrix mit Werten $\neq 1$, was einer uniformen Skalierung um den Faktor 5 entspricht.

   \item \textbf{(b) Translation} [Quelle: Folie 21]
         \[
            \begin{pmatrix}
               1 & 0 & 0 & 7 \\
               0 & 1 & 0 & 0 \\
               0 & 0 & 1 & 3 \\
               0 & 0 & 0 & 1
            \end{pmatrix}
         \]
         Die letzte Spalte enthält den Verschiebungsvektor $(7, 0, 3)^T$.

   \item \textbf{(c) Rotation} [Quelle: Folie 44]
         \[
            \begin{pmatrix}
               -1 & 0 & 0  & 0 \\
               0  & 1 & 0  & 0 \\
               0  & 0 & -1 & 0 \\
               0  & 0 & 0  & 1
            \end{pmatrix}
         \]
         Dies entspricht einer Rotation um die y-Achse um $180^\circ$ ($\cos 180^\circ = -1, \sin 180^\circ = 0$).

   \item \textbf{(d) Scherung} [Quelle: Folie 39]
         \[
            \begin{pmatrix}
               1 & 0 & 0 & 0 \\
               7 & 1 & 0 & 0 \\
               0 & 0 & 1 & 0 \\
               0 & 0 & 0 & 1
            \end{pmatrix}
         \]
         Das Element in der zweiten Zeile, erste Spalte ($a_{21}=7$) bewirkt eine Scherung der y-Koordinate in Abhängigkeit von x ($y' = y + 7x$).
\end{itemize}

\subsection*{c) Berechnung der Transformationsmatrix (Quelle: Folien 27, 37, 44)}
Gegeben: Punkt $P = (4, 0, 1)$. In homogenen Koordinaten: $P_h = (4, 0, 1, 1)^T$.
\textbf{1. Transformationsmatrizen:}
\begin{itemize}
   \item Skalierung ($S$) um Faktor 2:
         \[ S = \begin{pmatrix} 2 & 0 & 0 & 0 \\ 0 & 2 & 0 & 0 \\ 0 & 0 & 2 & 0 \\ 0 & 0 & 0 & 1 \end{pmatrix} \]
   \item Rotation ($R_x$) 90° um x-Achse ($\cos 90^\circ=0, \sin 90^\circ=1$):
         \[ R_x = \begin{pmatrix} 1 & 0 & 0 & 0 \\ 0 & 0 & -1 & 0 \\ 0 & 1 & 0 & 0 \\ 0 & 0 & 0 & 1 \end{pmatrix} \]
   \item Translation ($T$) um Vektor $(5, 0, 0)$:
         \[ T = \begin{pmatrix} 1 & 0 & 0 & 5 \\ 0 & 1 & 0 & 0 \\ 0 & 0 & 1 & 0 \\ 0 & 0 & 0 & 1 \end{pmatrix} \]
\end{itemize}
\textbf{2. Gesamttansformation und Anwendung:}
Reihenfolge: Erst Skalieren, dann Rotieren, dann Verschieben ($M = T \cdot R_x \cdot S$).

\[
   P' = T \cdot (R_x \cdot (S \cdot P_h))
\]

1. Skalierung:
\[ \begin{pmatrix} 2 & 0 & 0 & 0 \\ 0 & 2 & 0 & 0 \\ 0 & 0 & 2 & 0 \\ 0 & 0 & 0 & 1 \end{pmatrix} \begin{pmatrix} 4 \\ 0 \\ 1 \\ 1 \end{pmatrix} = \begin{pmatrix} 8 \\ 0 \\ 2 \\ 1 \end{pmatrix} \]

2. Rotation:
\[ \begin{pmatrix} 1 & 0 & 0 & 0 \\ 0 & 0 & -1 & 0 \\ 0 & 1 & 0 & 0 \\ 0 & 0 & 0 & 1 \end{pmatrix} \begin{pmatrix} 8 \\ 0 \\ 2 \\ 1 \end{pmatrix} = \begin{pmatrix} 8 \\ -2 \\ 0 \\ 1 \end{pmatrix} \]

3. Translation:
\[ \begin{pmatrix} 1 & 0 & 0 & 5 \\ 0 & 1 & 0 & 0 \\ 0 & 0 & 1 & 0 \\ 0 & 0 & 0 & 1 \end{pmatrix} \begin{pmatrix} 8 \\ -2 \\ 0 \\ 1 \end{pmatrix} = \begin{pmatrix} 13 \\ -2 \\ 0 \\ 1 \end{pmatrix} \]

\textbf{Ergebnis:} Der transformierte Punkt ist $P' = (13, -2, 0)$.

\section*{Aufgabe 8.2: Projektionen (Quelle: Folien 54-59)}

\subsection*{a) Die 6 klassischen Projektionen (Folie 54)}
\begin{enumerate}
   \item Aufriss (Frontansicht)
   \item Kabinett-Projektion
   \item Kavalliersperspektive
   \item Allgemeine Parallelprojektion
   \item Isometrische Perspektive
   \item Zentralperspektive (Vogelperspektive)
\end{enumerate}

\subsection*{b) 4 Eigenschaften von projektiven Abbildungen (Folie 55)}
\begin{enumerate}
   \item Geraden werden auf Geraden abgebildet.
   \item Schnitte von Geraden bleiben erhalten.
   \item Flächen werden auf Flächen abgebildet.
   \item Reihenfolge von Punkten auf projektiven Geraden bleibt erhalten.
\end{enumerate}

\subsection*{c) Unterschiede: Perspektivische vs. Parallele Projektion (Folie 56-58)}
\begin{enumerate}
   \item \textbf{Projektionszentrum:} Bei der perspektivischen Projektion treffen sich die Projektionsstrahlen in einem Punkt (Augpunkt/COP). Bei der parallelen Projektion verlaufen alle Strahlen parallel (DOP).
   \item \textbf{Parallelität:} Bei der parallelen Projektion bleiben parallele Geraden im Raum auch im Bild parallel. Bei der perspektivischen Projektion ist dies nicht der Fall (Fluchtpunkte).
   \item \textbf{Größenverhältnisse:} Perspektivische Projektion verkleinert entfernte Objekte (natürlicher Seheindruck), parallele Projektion erhält Größenverhältnisse (keine Tiefenverkürzung).
\end{enumerate}

\subsection*{d) Bevorzugte Projektion in der Medizin (Folie 59)}
In der Medizin wird oft die \textbf{parallele Projektion} bevorzugt.
\\
\textbf{Grund:} Längen und Abstände bleiben vergleichbar und messbar. Perspektivische Verzerrungen würden Größenverhältnisse verfälschen, was für Diagnosen (z.B. Tumorgröße) problematisch wäre.

\section*{Aufgabe 8.3: 2D/3D-Interaktion (Quelle: Folien 72-77)}

\subsection*{a) Problem der 3D-Interaktion mit 2D-Geräten (Folie 72, 76)}
Eine Bewegung auf dem Bildschirm kann unendlich vielen Bewegungen im 3D-Raum entsprechen.

\subsection*{b) Ansätze zur Umsetzung (Folie 73)}
\begin{itemize}
   \item Multi-Window (Nutzung mehrerer Ansichten gleichzeitig, z.B. Draufsicht, Seitenansicht).
   \item Direktes 2D-Maus-Mapping (z.B. Mausbewegung steuert x/y, Tastatur steuert z).
   \item Manipulatoren (grafische Hilfselemente).
\end{itemize}

\subsection*{c) Hilfsmittel für 3D-Interaktion (Folie 74-77)}
Das zentrale Hilfsmittel sind \textbf{Manipulatoren} (auch 3D-Widgets genannt).
\\
\textbf{Erläuterung:} Dies sind grafische Repräsentationen (z.B. Pfeile, Ringe, Würfel), die direkt am selektierten Objekt angezeigt werden. Sie erlauben es dem Nutzer, spezifische Transformationen (z.B. Rotation nur um die rote Achse) durch ``Dragging'' eines bestimmten Teils des Manipulators (``Handle'') auszuführen. Dadurch wird die Mehrdeutigkeit der Mausbewegung aufgelöst, da die Interaktion auf eine Achse oder Ebene beschränkt wird.

\end{document}