\documentclass[11pt,a4paper]{article}
\title{\"Ubung 2}
\author{Niclas Kusenbach, 360227; }

\usepackage[T1]{fontenc}
\usepackage[utf8]{inputenc}
\usepackage[ngerman,english]{babel}
\usepackage{amsmath,amsfonts,amssymb}
\usepackage{booktabs}
\usepackage{graphicx}
\usepackage{array}
\usepackage{siunitx}
\usepackage{physics}
\usepackage{xcolor}
\usepackage{enumitem}
\setlist{nosep}

\begin{document}

\maketitle

\section*{Aufgabe 2.1: Wahrnehmung}

\subsection*{a) Model of Mind}

\subsubsection*{Eingabe (Wahrnehmung, Perception, Sensory)}
Der Mensch nimmt Informationen über seine Sinne auf. Dies geschieht durch:
\begin{itemize}
  \item das visuelle Untersystem für das Sehen,
  \item das akustische Untersystem für das Hören oder
  \item das haptische Untersystem für das Fühlen.
\end{itemize}

\subsubsection*{Entscheidung (Cognition, Decision)}
Die Entscheidung wird im Gehirn getroffen.

\subsubsection*{Ausgabe (Reaktion, Response, Action, Motor)}
Der Mensch reagiert auf eine Eingabe mit seinem Körper, entweder durch
\begin{itemize}
  \item das stimmliche (Artikulations-)Untersystem für das Sprechen oder
  \item das motorische Untersystem für körperliche Bewegung.
\end{itemize}

\subsection*{b) Anwendung}

\subsubsection*{Beispiel 1: Farbbenennung}
\textbf{Situation:} Eine Person soll die Farbe eines Wortes benennen (z. B. das Wort „Rot“, das in blauer Farbe geschrieben ist).\\
\textbf{Eingabe (Perception):}
Über das visuelle Untersystem werden Farbe und Form des Wortes wahrgenommen (optischer Reiz trifft auf die Retina; Signalverarbeitung in der Sehrinde).\\
\textbf{Entscheidung (Cognition):}
Das Gehirn erkennt einen Konflikt zwischen Wortbedeutung und Schriftfarbe und entscheidet bewusst, die Schriftfarbe zu nennen.\\
\textbf{Ausgabe (Response):}
Über das stimmliche Untersystem (Artikulation) erfolgt die Reaktion durch das Aussprechen der richtigen Farbe.

\subsubsection*{Beispiel 2: Reflex beim Autofahren}
\textbf{Situation:} Eine Person fährt Auto, und ein Ball rollt plötzlich auf die Straße.\\
\textbf{Eingabe (Perception):}
Das visuelle Untersystem nimmt die Bewegung des Balls wahr; zusätzlich wird über das auditive System (Geräusch) ein Reiz aufgenommen.\\
\textbf{Entscheidung (Cognition):}
Das Gehirn interpretiert den Reiz als Gefahrensituation und ruft gespeicherte Handlungsstrategien aus dem Langzeitgedächtnis ab.\\
\textbf{Ausgabe (Response):}
Über das motorische Untersystem erfolgt eine schnelle Reaktion (z. B. Bremsen).

\subsection*{c) Reiz und Wahrnehmung (vgl. Slides 26, 82, 83)}
Ein \textbf{Reiz} ist ein physikalisches Signal, das auf die Sinnesorgane wirkt (z. B. Licht- oder Schallwellen).  
Die \textbf{Wahrnehmung} ist die subjektive Interpretation dieses Reizes durch das Gehirn.\\
Daher gilt: \textbf{Reiz $\neq$ Wahrnehmung}, da Wahrnehmung ein konstruktiver Prozess ist, der von Kontext, Erfahrung und Aufmerksamkeit abhängt.

\section*{Aufgabe 2.2: Das Auge}

\subsection*{a) Stäbchen und Zäpfchen (vgl. Slides 30--47)}
Stäbchen sind sehr lichtempfindlich und reagieren bereits auf wenig Licht. Ihre Funktion liegt im Hell- und Dunkelsehen, und man findet sie hauptsächlich außerhalb der Fovea.  
Zäpfchen hingegen befinden sich vor allem in der Fovea. Sie werden erst bei hellem Licht aktiv, und mit ihrer Hilfe können wir Farben bei Tageslicht wahrnehmen.

\subsection*{b) Vier Arten von Zellen (vgl. Slide 48)}
\begin{itemize}
  \item Horizontale Zellen
  \item Amakrinzellen
  \item Bipolarzellen
  \item Ganglienzellen
\end{itemize}

\section*{Aufgabe 2.3: Depth Cues}

\subsection*{a) Drei Arten von Depth Cues (vgl. Slide 87)}
\begin{itemize}
  \item Binokulare Depth Cues
  \item Pictorial Depth Cues
  \item Dynamische Depth Cues
\end{itemize}

\subsection*{b) Anwendungen von Depth Cues}
\begin{enumerate}
  \item Film und Animation
  \item VR-Systeme
  \item Architektur
\end{enumerate}

\section*{Aufgabe 2.4: Vorverarbeitung visueller Informationen}

\subsection*{a) Brightness und Lightness}
„Brightness“ entspricht der wahrgenommenen Menge an Licht, das von einer selbstleuchtenden Lichtquelle ausgeht.\\
„Lightness“ entspricht der wahrgenommenen Menge an Licht, das von einer reflektierenden Oberfläche ausgeht (vgl. Slide 60).

\subsection*{b) Checker-Shadow-Illusion (vgl. \texttt{https://persci.mit.edu/gallery/checkershadow/description})}
Das Gehirn korrigiert die Farbwahrnehmung basierend auf Erfahrung. Da Schatten Objekte dunkler erscheinen lassen, interpretiert das visuelle System ein Feld im Schatten als heller, als es physikalisch ist. Das visuelle System verarbeitet keine isolierten Pixel, sondern Muster, Lichtquellen und Oberflächen.

\section*{Aufgabe 2.5: Aufmerksamkeit und Gedächtnis}

\subsection*{a) Filter im Gehirn (vgl. Slide 117)}
\begin{itemize}
  \item \textbf{Gewählt (selective):} Das Gehirn fokussiert auf relevante Informationen.
  \item \textbf{Geteilt (divided):} Das Gehirn versucht, mehrere Aufgaben gleichzeitig auszuführen, was die Verarbeitung verlangsamen kann.
  \item \textbf{Erfasst (captured):} Ein Reiz zieht die gesamte Aufmerksamkeit auf sich.
\end{itemize}

\subsection*{b) Aufmerksamkeit und Gedächtnis}
Das \textbf{Arbeitsgedächtnis} speichert Informationen kurzfristig (ca. 200~ms) und erlaubt schnellen Zugriff (ca. 70~ms).  
Das \textbf{Langzeitgedächtnis} hat eine nahezu unbegrenzte Kapazität, kann Informationen speichern, abrufen und vergessen, wenn ihnen keine Bedeutung beigemessen wird.

\end{document}