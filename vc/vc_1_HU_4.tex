\documentclass[11pt,a4paper]{article}
\title{\"Ubung 4: Gruppe 28}
\author{Niclas Kusenbach, 360227 \and Alicia Bayerl, 2633336 \and Mohamed Naceur Hedhili, 2957151 \and Selma Naz Öner, 2662640}

\usepackage[T1]{fontenc}
\usepackage[utf8]{inputenc}
\usepackage[ngerman,english]{babel}
\usepackage{amsmath,amsfonts,amssymb}
\usepackage{booktabs}
\usepackage{graphicx}
\usepackage{array}
\usepackage{siunitx}
\usepackage{physics}
\usepackage{xcolor}
\usepackage{enumitem}
\usepackage{hyperref}
\usepackage{tabularx}
\usepackage{tikz}
\setlist{nosep}

\begin{document}

\maketitle

\section*{Aufgabe 4.1: Quiz}

\subsection*{a) (vgl. Slide 23)}
\begin{itemize}
  \item \textbf{Falsch:} Die Anzahl der Unstetigkeiten innerhalb einer Periode ist unendlich. \par
        $\rightarrow$ Es gibt eine endliche Anzahl von Unstetigkeiten innerhalb einer Periode.
  \item \textbf{Richtig:} Die Anzahl der Maxima und Minima innerhalb einer Periode ist endlich.
  \item \textbf{Falsch:} Die Funktion ist nicht in jeder Periode integrierbar.\par
        $\rightarrow$ Die Funktion ist in jeder Periode integrierbar (d.h., die Fläche unter dem Betrag der Funktion ist in jeder Periode endlich)
\end{itemize}
\subsection*{b) Fourier Darstellung (vgl. Slide 61)}
Die Fourier Darstellung ist die Zerlegung einer Funktion in ihre Frequenzbestandteile.

\subsection*{c) (vgl. Slide 70)}
Einer Faltung im \textbf{Ortsraum} entspricht einer \textbf{Multiplikation} im Frequenzraum.

\section*{4.2: Fourier Reihen}

\subsection*{a) Theoretischer Teil: Symmetrieeigenschaften}

\subsubsection*{Bestimmten Sie, wann eine Funktion gerade und ungerade ist. (vgl. Slide 44)}
Gerade Funktion: $f(-t) = f(t)$\\
Ungerade Funktion: $f(-t) = -f(t)$

\subsubsection*{Geben Sie an, wie sich die Koeffizienten $a_n$ (für $n \geq 0$) und $b_n$ (für $n \geq 1$) für gerade Funktionen vereinfachen. (vgl. Slide 35, 37, 45)}
Für gerade Funktionen sind alle $b_n = 0 \text{ und } a_n = \frac{2}{\pi}\int_0^\pi f(x)\cos(nx)\,dx \qquad n \ge 0 $

\subsubsection*{Geben Sie an, wie sich die Koeffizienten $a_n$ (für $n \geq 0$) und $b_n$ (für $n \geq 1$) für ungerade Funktionen vereinfachen. (vgl. Slide 35, 37, 45)}
Für ungerade Funktionen sind alle $a_n = 0$, $b_n = \frac{2}{\pi} \int_{0}^{\pi} f(x)\sin(nx)\,dx \qquad (n \ge 1)$.

\subsection*{b) Praktischer Teil: Anwendung der Symmetrie}

\textbf{Gegeben:} \( f(x) = x \) im Intervall \([-\pi, \pi]\)\\
\textbf{Gesucht:} $ a_0,\; a_n,\; b_n \text{ für } n \ge 1 $

\subsubsection*{Symmetrieprüfung}
$f(x) = x$ ist ungerade, da
\[
  f(-x) = -x = -f(x)
\]
Deshalb vereinfachen sich die Fourierkoeffizienten auf:
\[
  a_0 = 0, \quad a_n = 0.
\]

\subsubsection*{Berechnung von \(b_n\)}

\[
  b_n = \frac{1}{\pi} \int_{-\pi}^{\pi} f(x) \sin(nx)\, dx
\]
Da $f(x)$ und $\sin(nx)$ ungerade sind, ist das Produkt grade. Vereinfacht:
\[
  b_n = \frac{2}{\pi} \int_{0}^{\pi} x \sin(nx)\, dx
\]

\subsubsection*{Partielle Integration}

\[
  u = x \Rightarrow u' = 1 \\
  v' = sin(nx) \Rightarrow v = -\frac{1}{n}\cos(nx)
\]

\[
  \int x \sin(nx)dx =  \left[ x\left( -\frac{1}{n}\cos(nx)\right) \right] - \int 1 \left( -\frac{1}{n}\cos(nx)\right)dx
\]
\begin{enumerate}
  \item $-\frac{x}{n}\cos(nx)$\vspace{6pt}
  \item $\frac{1}{n} \int \cos(nx)dx = \frac{1}{n}\cdot\frac{1}{n}\sin(nx) = \frac{1}{n^2}\sin(nx)$
\end{enumerate}

\[
  \begin{aligned}
    b_n & = \frac{2}{\pi}\left[ -\frac{x}{n}\cos(nx) + \frac{1}          {n^2}\sin(nx)\right]_{0}^{\pi}                                               \\
        & = \frac{2}{\pi}\left( -\frac{\pi}{n}\cos(n\pi) + \frac{1}{n^2}\sin(n\pi) - \left( -\frac{0}{n}\cos(0) + \frac{1}{n^2}\sin(0) \right)\right) \\
        & = \frac{2}{\pi}\left( -\frac{\pi}{n}\cos(n\pi) \right)                                                                                      \\
        & = -\frac{2}{n}\cos(n\pi)
  \end{aligned}
\]
Da $\cos(n\pi) = (-1)^n$,
\[
  b_n = -\frac{2}{n}(-1)^n = \frac{2}{n}(-1)^{n+1}
\]

\subsubsection*{Ergebnis}

Die Fourier-Reihe lautet somit:
\[
  \begin{aligned}
    f(x) & = 0 + \sum_{n=1}^{\infty}\left( 0\cos(nx) + \frac{2}{n}(-1)^{n+1}\sin(nx)\right) \\
         & =\sum_{n=1}^{\infty}\frac{2}{n}(-1)^{n+1}\sin(nx)
  \end{aligned}
\]

\section*{4.3: Faltung und Filterung}

\subsection*{a) Suche über Raum und Skalierung (vgl. Slide 68)}

Der Ansatz heißt Sliding-Window-Approach.

\subsection*{b) Schritte (vgl. Slide 68)}

\begin{enumerate}
  \item Ein Eingabebild wird in Ein-Pixel-Schritten
        horizontal und vertikal gescannt
  \item Das Bild wird um den Faktor 1,2
        verkleinert, die Suche wiederholt
  \item Wiederholung
\end{enumerate}

\section*{4.4: Komplexe Zahlen}

\[
  z_1 = 7 - 3i,\quad z_2 = 7 + 5i
\]

\[
  \begin{aligned}
    z_3 & = z_1 \cdot z_2                                 \\
        & = (7 - 3i)(7 + 5i)                              \\
        & = 7\cdot 7 + 7\cdot 5i - 3i\cdot 7 - 3i\cdot 5i \\
        & = 49 + 35i - 21i - 15i^2                        \\
        & = 49 + 14i -15i^2
  \end{aligned}
\]

Da \( i^2 = -1 \), gilt:

\[
  \begin{aligned}
    z_3 & = 49 + 14i - 15 \cdot (-1) \\
        & = 49 + 14i + 15            \\
        & = 64 + 14i
  \end{aligned}
\]

\begin{tikzpicture}[scale=0.8]
  % Achsen
  \draw[->,black,line width=1pt] (-1,0) -- (10,0);
  \node at (10.3,0.3) {Re};
  \draw[->,black,line width=1pt] (0,-4) -- (0,6);
  \node at (0.3,6.3) {Im};

  % Nur x = 0, 5, 10
  \foreach \x in {5,10} {
      \node[below] at (\x,0) {\x};
    }

  % Nur y = -4, 0, 6
  \foreach \y in {-4,6} {
      \node[left] at (0,\y) {\y};
    }


  % Gitterlinien optional
  \draw[black!40,line width=1pt] (0,-3) -- (7,-3);
  \draw[black!40,line width=1pt] (0,5) -- (7,5);
  \draw[step=1cm,black!10] (-1,-5) grid (11,7);

  % Punkte
  \filldraw[blue] (7,-3) circle (2pt) node[right] {$z_1 = 7-3i$};
  \filldraw[red]  (7,5)  circle (2pt) node[right] {$z_2 = 7+5i$};

  % Markierungen
  \draw[dashed] (7,0) -- (7,-3);
  \draw[dashed] (7,0) -- (7,5);

  % Nullpunkt Beschriftung
  \node at (-0.3,-0.3) {$0$};
\end{tikzpicture}

\end{document}