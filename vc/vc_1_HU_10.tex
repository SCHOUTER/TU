\documentclass[11pt,a4paper]{article}

% ---------------------------------------------------------
% Encoding, Fonts, Typography
% ---------------------------------------------------------
\usepackage[utf8]{inputenc}
\usepackage[T1]{fontenc}
\usepackage{lmodern}
\usepackage{microtype}

% ---------------------------------------------------------
% Language
% ---------------------------------------------------------
\usepackage[ngerman,english]{babel}

% ---------------------------------------------------------
% Math
% ---------------------------------------------------------
\usepackage{amsmath, amssymb, amsfonts}
\usepackage{mathtools}

% ---------------------------------------------------------
% Tables
% ---------------------------------------------------------
\usepackage{booktabs}
\usepackage{tabularx}

% ---------------------------------------------------------
% Figures & Graphics
% ---------------------------------------------------------
\usepackage{graphicx}
\usepackage{tikz}
\usetikzlibrary{shapes.geometric}


% ---------------------------------------------------------
% Lists
% ---------------------------------------------------------
\usepackage{enumitem}
\setlist{nosep}

% ---------------------------------------------------------
% Units & Numbers
% ---------------------------------------------------------
\usepackage{siunitx}

% ---------------------------------------------------------
% Hyperlinks
% ---------------------------------------------------------
\usepackage[hidelinks]{hyperref}
\usepackage{xurl}

% ---------------------------------------------------------
% Global layout improvements
% ---------------------------------------------------------
\clubpenalty=10000
\widowpenalty=10000
\emergencystretch=3em

% ---------------------------------------------------------
% Title
% ---------------------------------------------------------
\title{Übung 10: Gruppe 28}
\author{
    Niclas Kusenbach, 360227 \and
    Alicia Bayerl, 2633336
}

\begin{document}
\maketitle

\section*{Aufgabe 10.1: Szenengraphstruktur (3P)}

\subsection*{a) Vier Eigenschaften eines Szenengraphen (2P)}

\begin{enumerate}
  \item \textbf{Gerichtet:} Die Kanten im Graphen haben eine definierte Richtung, typischerweise von einem Elternknoten zu einem oder mehreren Kindknoten.
  \item \textbf{Azyklisch:} Der Graph darf keine geschlossenen Schleifen enthalten. Dies verhindert Endlosschleifen beim Rendering.
  \item \textbf{Wurzelknoten:} Es existiert genau ein ausgezeichneter Startknoten (die Wurzel), von dem aus der gesamte Graph traversiert werden kann.
  \item \textbf{Hierarchisch:} Der Graph organisiert Objekte in einer logischen oder räumlichen Struktur (z. B. „Rad ist Teil des Autos“). Transformationen von Elternknoten werden auf Kindknoten vererbt.
\end{enumerate}

\subsection*{b) Drei Vorteile des Szenengraph-Konzeptes (1P)}

\begin{itemize}
  \item \textbf{Semantische Gruppierung:} Komplexe Szenen lassen sich verwalten, indem zusammengehörige Objekte (z. B. alle Teile eines Motorrollers) gruppiert werden. Dadurch können sie referenziert werden und müssen nur einmal im Speicher liegen.
  \item \textbf{Hierarchische Transformationen:} Bewegt man einen Elternknoten, bewegen sich alle Kindknoten automatisch mit. Dies erleichtert Animationen (z. B. dreht sich der Lenker, bewegt sich das Vorderrad mit).
  \item \textbf{Wiederverwendung:} Ganze Äste des Baums können vom Rendering ausgeschlossen werden, wenn der Elternknoten nicht sichtbar ist (Culling). Zudem können Geometrien durch Instanziierung speicherschonend wiederverwendet werden.
\end{itemize}

\vspace{1cm}

\section*{Aufgabe 10.2: Szenengraph (3P)}

\vspace{0.5cm}

\begin{center}
  \includegraphics[width=0.7\textwidth]{vc_1_HU_10_1.png}
\end{center}

\section*{Aufgabe 10.3: Anwendungsaufgabe (3P)}

\begin{center}
  \includegraphics[width=0.7\textwidth]{vc_1_HU_10.png}
\end{center}

\end{document}