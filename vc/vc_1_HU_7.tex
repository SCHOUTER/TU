\documentclass[11pt,a4paper]{article}

% ---------------------------------------------------------
% Encoding, Fonts, Typography
% ---------------------------------------------------------
\usepackage[utf8]{inputenc}
\usepackage[T1]{fontenc}
\usepackage{lmodern}
\usepackage{microtype}

% ---------------------------------------------------------
% Language
% ---------------------------------------------------------
\usepackage[ngerman,english]{babel}

% ---------------------------------------------------------
% Math
% ---------------------------------------------------------
\usepackage{amsmath, amssymb, amsfonts}
\usepackage{mathtools}

% ---------------------------------------------------------
% Tables
% ---------------------------------------------------------
\usepackage{booktabs}
\usepackage{tabularx}

% ---------------------------------------------------------
% Figures & Graphics
% ---------------------------------------------------------
\usepackage{graphicx}
\usepackage{tikz}

% ---------------------------------------------------------
% Lists
% ---------------------------------------------------------
\usepackage{enumitem}
\setlist{nosep}

% ---------------------------------------------------------
% Units & Numbers
% ---------------------------------------------------------
\usepackage{siunitx}

% ---------------------------------------------------------
% Hyperlinks
% ---------------------------------------------------------
\usepackage[hidelinks]{hyperref}
\usepackage{xurl}

% ---------------------------------------------------------
% Global layout improvements
% ---------------------------------------------------------
\clubpenalty=10000
\widowpenalty=10000
\emergencystretch=3em

% ---------------------------------------------------------
% Title
% ---------------------------------------------------------
\title{\"Ubung 6: Gruppe 28}
\author{
    Niclas Kusenbach, 360227 \and
    Alicia Bayerl, 2633336 \and
    Mohamed Naceur Hedhili, 2957151 \and
    Selma Naz Öner, 2662640
}

\begin{document}
\maketitle

\section*{7.1 Schatten}

\subsection*{a) Phong-Beleuchtungsmodell (vgl. Slide 66)}
Die Lichtintensität $I_{total}$ an einem Punkt setzt sich zusammen aus:
\[ I_{total} = I_{amb} + I_{diff} + I_{spec} \]

\begin{enumerate}
  \item \textbf{Ambientes Licht ($I_{amb}$):} Grundhelligkeit der Szene (indirekte Beleuchtung simuliert durch Konstante). Richtungsunabhängig.
        \[ I_{amb} = k_{amb} \cdot C_{amb} \]

  \item \textbf{Diffuse Reflexion ($I_{diff}$):} Matte Oberflächen (Lambert-Reflexion). Helligkeit hängt vom Winkel zwischen Lichtvektor $L$ und Oberflächennormale $N$ ab.
        \[ I_{diff} = k_{diff} \cdot C_{light} \cdot (\vec{N} \cdot \vec{L}) \]

  \item \textbf{Spiegelnde Reflexion ($I_{spec}$):} Glanzpunkte (Highlights). Hängt vom Winkel zwischen Reflexionsvektor $R$ und Blickvektor $V$ ab. Der Exponent $m$ bestimmt die Rauhigkeit (je höher $m$, desto kleiner und schärfer der Glanzpunkt).
        \[ I_{spec} = k_{spec} \cdot C_{light} \cdot (\vec{R} \cdot \vec{V})^m \]
\end{enumerate}

\subsection*{b) weitere Schatten (vgl. Slide 75)}

\begin{enumerate}
  \item Self-Shadowing: Wenn ein Objekt Schatten auf sich selbst wirft.
  \item Cast Shadows: Der Schatten, den ein Objekt auf andere Objekte wirft.
\end{enumerate}

\section*{7.2 Rasterisierung (vgl. Slide 92)}

Gegeben: Startpunkt $(x_1, y_1) = (1, 2)$, Endpunkt $(x_2, y_2) = (5, 7)$.
Berechnung der Differenzen: $\Delta x = 5 - 1 = 4$, $\Delta y = 7 - 2 = 5$ \newline
Da $\Delta y > \Delta x$, handelt es sich um eine steile Linie (Steigung > 1). Der Algorithmus muss daher entlang der y-Achse schreiten und die x-Koordinate anpassen. Wir vertauschen die Logik von x und y im Vergleich zum Standard-Algorithmus auf Folie 92.

\begin{align*}
  1. & \quad \delta x \leftarrow 7 - 2 = 5 \quad (\text{entspricht real } \Delta y)                                  \\
  2. & \quad \delta y \leftarrow 5 - 1 = 4 \quad (\text{entspricht real } \Delta x)                                  \\
  3. & \quad x \leftarrow 2                                                                                          \\
  4. & \quad y \leftarrow 1                                                                                          \\
  5. & \quad \textbf{Setze Pixel} (y, x) \Rightarrow (1, 2)                                                          \\
  6. & \quad \xi \leftarrow \lfloor \delta x / 2 \rfloor = \lfloor 5 / 2 \rfloor = 2 \quad (\text{Integer-Division})
\end{align*}
Die Schleife läuft solange $x < x_2$ (also $x < 7$).

\begin{table}[h!]
  \centering
  \renewcommand{\arraystretch}{1.3}
  \resizebox{\textwidth}{!}{
    \begin{tabular}{|c|c|l|c|l|l|c|}
      \hline
      \textbf{Iter.} & \textbf{x (Start)} & \textbf{Schritt 9: $\xi \leftarrow \xi - \delta y$} & \textbf{Test $\xi < 0$} & \textbf{y (neu)} & \textbf{Schritt 12: $\xi$ (Korr.)} & \textbf{Pixel $(y, x)$} \\
      \hline
      \hline
      1              & $2 \to 3$          & $2 - 4 = -2$                                        & \textbf{Ja}             & $1 + 1 = 2$      & $-2 + 5 = 3$                       & (2, 3)                  \\
      \hline
      2              & $3 \to 4$          & $3 - 4 = -1$                                        & \textbf{Ja}             & $2 + 1 = 3$      & $-1 + 5 = 4$                       & (3, 4)                  \\
      \hline
      3              & $4 \to 5$          & $4 - 4 = 0$                                         & \textbf{Nein}           & $3$ (bleibt)     & $-$ (keine Korr.)                  & (3, 5)                  \\
      \hline
      4              & $5 \to 6$          & $0 - 4 = -4$                                        & \textbf{Ja}             & $3 + 1 = 4$      & $-4 + 5 = 1$                       & (4, 6)                  \\
      \hline
      5              & $6 \to 7$          & $1 - 4 = -3$                                        & \textbf{Ja}             & $4 + 1 = 5$      & $-3 + 5 = 2$                       & (5, 7)                  \\
      \hline
    \end{tabular}
  }
\end{table}

\begin{center}
  \includegraphics[width=0.7\linewidth]{vc_1_HU_7.jpeg}
\end{center}

\section{7.3 Räumliche Datenstrukturen}

\subsection*{a) Binary Space Partitioning (vgl. Slide 61, Moodle Forum)}

\textbf{Theorie: Definition der Unterteilungsebenen}
\begin{itemize}
  \item Die Schnittebenen werden in einem BSP-Tree üblicherweise durch die Polygone der Szene selbst definiert (sogenanntes Auto-Partitioning).
  \item \textit{Zusatzinfo aus Forum:} Der Unterteilungsprozess wird fortgesetzt, bis alle resultierenden Regionen (Zellen) konvex sind. Da einzelne Liniensegmente/Polygone stets konvex sind, endet der Prozess spätestens, wenn jeder Blattknoten nur noch ein Fragment enthält.
\end{itemize}

\textbf{Anwendung: Fortführung des Algorithmus}

\begin{center}
  \includegraphics[width=0.7\linewidth]{vc_1_HU_7_1.png}
  \includegraphics[width=0.7\linewidth]{vc_1_HU_7_2.png}
\end{center}

\subsection*{b) Backface Culling (vgl. Slide 84)}

Gegeben: $\vec{n} =
  \begin{pmatrix}
    0 \\
    0 \\
    1
  \end{pmatrix}$, $\vec{s} =
  \begin{pmatrix}
    0 \\
    1 \\
    1
  \end{pmatrix}$ \vspace{10pt}\newline
Aus der Vorlesung wissen wir, dass wenn $\vec{n} \cdot \vec{s} > 0$, dann ist es eine Rückseite. Also rechnen wir:

\[\vec{n} \cdot \vec{s} = \begin{pmatrix}
    0 \\
    0 \\
    1
  \end{pmatrix} \cdot \begin{pmatrix}
    0 \\
    1 \\
    1
  \end{pmatrix}
  =
  0 \cdot 0 + 0 \cdot 1 + 1 \cdot 1 = 1
\]\vspace{10pt}\newline
Damit ist $\vec{n} \cdot \vec{s} = 1 > 0$ und das Dreieck mit der Normalen $\vec{n}$ ist eine Rückseite.

\subsection*{c) (vgl. Slide 83, 98)}
Bei konvexen Körpern entfernt Backface Culling alle unsichtbaren Flächen und macht damit z-Buffer theoretisch überflüssig.

Ein Nachteil ist, dass Transparenz nicht korrekt realisierbar ist, da der z-Buffer nur einen Tiefenwert pro Pixel speichert. Es kann immer nur das vorderste Objekt dargestellt werden, wodurch dahinterliegende Objekte verworfen werden.


\end{document}