\documentclass[
../../vc_summary.tex,
]
{subfiles}

\externaldocument[ext:]{../../vc_summary}
% Set Graphics Path, so pictures load correctly
\graphicspath{{../}}

\begin{document}

\section{Bilder}

Ein digitales Bild entsteht durch die Diskretisierung einer kontinuierlichen Szene.
\begin{itemize}
  \item \textbf{Lochkameramodell:} Projektion der 3D-Welt auf eine 2D-Ebene.
  \item \textbf{Rasterisierung:} Umwandlung des kontinuierlichen Signals in ein digitales Gitter (Pixel).
  \item \textbf{Repräsentation:} Ein Bild wird als Matrix von Intensitätswerten (Grauwerten oder Farbkanälen) dargestellt.
\end{itemize}

\subsection{Digitale Bildverarbeitung im Ortsraum}

Im Ortsraum (Spatial Domain) werden Operationen direkt auf den Pixelwerten des Bildes durchgeführt.

\subsubsection{Pixeloperationen}
Pixeloperationen manipulieren einen Pixelwert unabhängig von seiner Nachbarschaft (globaler Kontext, aber lokale Anwendung).

\begin{defbox}[Grauwert-Abbildung (Mapping)]
  Die Transformation eines Pixels an der Stelle $(m,n)$ wird beschrieben durch:
  $$ g[m,n] = T(f[m,n]) $$
  wobei $f$ das Eingabebild, $g$ das Ausgabebild und $T$ die Transformationsfunktion ist.
\end{defbox}

\textbf{Wichtige Pixeloperationen:}
\begin{itemize}
  \item \textbf{Negativ:} $g[m,n] = f_{max} - f[m,n]$ (Invertierung der Helligkeit).
  \item \textbf{Binärisierung (Thresholding):}
        $$ g[m,n] = \begin{cases} f_{max} & \text{falls } f[m,n] > \tau \\ f_{min} & \text{falls } f[m,n] \le \tau \end{cases} $$
        Dient zur Trennung von Vorder- und Hintergrund.
  \item \textbf{Grauwertfensterung:} Hervorheben eines bestimmten Intensitätsintervalls (z.B. in der Medizintechnik/CT).
  \item \textbf{Kontrastspreizung:} Abbildung des genutzten Grauwertbereichs auf die volle Skala (z.B. 0-255), um den visuellen Kontrast zu erhöhen.
  \item \textbf{Mittelung:} Unterdrückung von unkorreliertem Rauschen durch Mittelung über mehrere Aufnahmen derselben Szene ($g = \frac{1}{k}\sum f_i$).
\end{itemize}

\subsubsection{Histogramme}
Ein Histogramm ist die graphische Darstellung der Häufigkeitsverteilung der Grauwerte in einem Bild.

\begin{itemize}
  \item \textbf{Bildhelligkeit:} Entspricht dem Mittelwert aller Grauwerte.
  \item \textbf{Bildkontrast:} Entspricht der Varianz aller Grauwerte.
\end{itemize}

% [HIER GRAFIK EINFÜGEN: Beispiele für Histogramme (dunkel, hell, kontrastarm, kontrastreich)]

\textbf{Histogrammausgleich (Equalization):}
Ziel ist es, die Grauwerte so umzuverteilen, dass sie gleichmäßig über den gesamten Bereich verteilt sind (Maximierung des Kontrasts). Dies geschieht über die \defc{Summenwahrscheinlichkeit} (kumulative Verteilungsfunktion).
$$ p(g) = \max(\text{Intensität}) \cdot \sum_{i=0}^{g} p(i) $$
Dies ist eine verlustbehaftete Operation und nicht umkehrbar.

\subsubsection{Filterung im Ortsraum (Konvolution)}
Hierbei wird der neue Wert eines Pixels unter Berücksichtigung seiner Nachbarschaft berechnet (lokaler Kontext).

\begin{defbox}[Faltung (Convolution)]
  Die Faltung eines Bildes $f$ mit einer Filtermaske $w$ (Kernel) ist definiert als:
  $$ (f * w)(m,n) = \sum_{i=-k/2}^{k/2} \sum_{j=-l/2}^{l/2} w(i,j) \cdot f(m+i, n+j) $$
\end{defbox}

\paragraph{Tiefpass-Filter (Glättung)}
Dienen der Rauschunterdrückung und Weichzeichnung (Blurring).
\begin{itemize}
  \item \textbf{Eigenschaften:} Koeffizienten sind positiv und summieren sich meist zu 1 (Normalisierung), um die Helligkeit zu erhalten.
  \item \textbf{Box-Filter (Mittelwert):} Alle Koeffizienten sind gleich (z.B. alle $\frac{1}{9}$ bei einer $3\times3$ Matrix). Nachteil: Erzeugt Artefakte (rechteckige Unschärfe).
  \item \textbf{Gauß-Filter:} Approximation der Gauß-Glocke.
        $$ G(x,y) = \frac{1}{2\pi\sigma^2} e^{-\frac{x^2+y^2}{2\sigma^2}} $$
        Gewichtet zentrale Pixel stärker als äußere. Besser als Box-Filter, da natürliche Glättung. Kann durch Binomialfilter approximiert werden.
\end{itemize}

\paragraph{Nicht-Lineare Filter}
\begin{itemize}
  \item \textbf{Median-Filter:} Ersetzt den Pixelwert durch den \defc{Median} der Nachbarschaft.
  \item \textbf{Vorteil:} Entfernt "Salt-and-Pepper"-Rauschen (Ausreißer) extrem effektiv und \defc{erhält Kanten} (im Gegensatz zu Tiefpassfiltern, die Kanten verschmieren).
\end{itemize}

\paragraph{Hochpass-Filter (Kantenextraktion)}
Dienen dem Finden von Kanten (Hervorhebung von hohen Frequenzen/Änderungen).
\begin{itemize}
  \item \textbf{Eigenschaften:} Koeffizienten sind positiv und negativ, Summe ist meist 0. Ergebnis kann negative Werte enthalten.
  \item \textbf{Gradienten (1. Ableitung):} Reagieren auf Rampen und Stufen (Kanten).
  \item \textbf{Laplace-Filter (2. Ableitung):} Isotrop (richtungsunabhängig).
        $$ \nabla^2 f(x,y) = \frac{\partial^2 f}{\partial x^2} + \frac{\partial^2 f}{\partial y^2} $$
        Reagiert stark auf feine Details und Rauschen ("Double Response" an Kanten: Nulldurchgang).
  \item \textbf{Laplacian of Gaussian (LoG):} Auch "Mexican Hat" oder Sombrero-Filter. Kombination aus Glättung (Gauß) zur Rauschunterdrückung und Laplace zur Kantenfindung.
\end{itemize}

\subsection{Digitale Bildverarbeitung im Frequenzraum}

Bilder können durch Transformationen (z.B. Fourier) in Frequenzanteile zerlegt werden.
\begin{itemize}
  \item \textbf{Tiefe Frequenzen:} Grobe Strukturen, Flächen, homogener Hintergrund. (Im Zentrum des Spektrums).
  \item \textbf{Hohe Frequenzen:} Feine Details, Kanten, Rauschen. (Außen im Spektrum).
\end{itemize}

\begin{defbox}[Faltungssatz]
  Eine Faltung im Ortsraum entspricht einer Multiplikation im Frequenzraum (und umgekehrt).
  $$ f * h = F^{-1}( F(f) \cdot F(h) ) $$
\end{defbox}

\subsubsection{Ablauf der Frequenzfilterung}
1. Transformation des Eingabebildes $f$ mittels FFT (Fast Fourier Transform) $\rightarrow F(u,v)$.
2. Multiplikation mit der Filterfunktion $H(u,v)$.
3. Rücktransformation mittels inverser FFT $\rightarrow$ gefiltertes Bild $g(x,y)$.

\subsubsection{Filtertypen im Frequenzraum}
\begin{itemize}
  \item \textbf{Idealer Tiefpass:} Schneidet alle Frequenzen oberhalb $D_0$ ab (Rechteckfunktion). Führt im Ortsraum zu \defc{Ringing-Artefakten} (Wellen um Kanten), da die FT einer Rechteckfunktion eine Sinc-Funktion ist.
  \item \textbf{Gauß-Tiefpass:} Weicher Übergang. Kein Ringing, da FT einer Gauß-Funktion wieder eine Gauß-Funktion ist.
  \item \textbf{Hochpass:} Kehrt das Prinzip um (Zentrum wird auf 0 gesetzt, hohe Frequenzen bleiben).
\end{itemize}

\textbf{Aliasing:} Tritt auf, wenn Abtastrate zu gering (Unterabtastung) oder im Frequenzraum Frequenzen abgeschnitten werden ("Leakage").

\subsection{Bildkompression}

Ziel: Reduktion der Datenmenge durch Eliminierung von Redundanzen (räumlich, zeitlich, spektral) und Irrelevanzen (nicht wahrnehmbare Informationen).

\subsubsection{Klassifikation}
\begin{itemize}
  \item \textbf{Verlustfrei (Lossless):} Original kann exakt wiederhergestellt werden. (z.B. Huffman, RLE, LZW $\to$ PNG, GIF, TIFF).
  \item \textbf{Verlustbehaftet (Lossy):} Informationen gehen verloren, höhere Kompressionsraten. Nutzt Wahrnehmungsmodelle des Menschen (z.B. JPEG, MPEG).
\end{itemize}

\subsubsection{JPEG (Joint Photographic Experts Group)}
Ein Standard für verlustbehaftete Kompression, optimiert für natürliche Fotos (fließende Übergänge).

\textbf{Die 5 Schritte des JPEG-Baseline-Codecs:}

\begin{enumerate}
  \item \textbf{Farbraum-Transformation:}
        Umwandlung von RGB in $YC_bC_r$.
        \begin{itemize}
          \item $Y$: Luminanz (Helligkeit).
          \item $C_b, C_r$: Chrominanz (Farbdifferenzen Blau/Rot).
        \end{itemize}

  \item \textbf{Farb-Subsampling (Downsampling):}
        Da das menschliche Auge Helligkeit besser auflöst als Farbe, werden $C_b$ und $C_r$ räumlich reduziert (z.B. 4:2:0 oder 2x2 Mittelung). $Y$ bleibt voll erhalten.

  \item \textbf{Block-Aufteilung \& DCT (Diskrete Kosinustransformation):}
        Das Bild wird in $8\times8$ Blöcke unterteilt. Jeder Block wird mittels DCT in den Frequenzraum transformiert.
        \begin{itemize}
          \item Ergebnis: 1 DC-Koeffizient (Gleichanteil/Grundhelligkeit) und 63 AC-Koeffizienten (Wechselanteile).
          \item Energiekonzentration: Meist sind nur wenige Koeffizienten (niedrige Frequenzen) signifikant groß.
        \end{itemize}

  \item \textbf{Quantisierung (Der eigentliche Verlust):}
        $$ F^Q = \text{Round}( F_{ij} / Q_{ij} ) $$
        Die DCT-Koeffizienten werden durch eine Quantisierungsmatrix $Q$ geteilt und gerundet.
        \begin{itemize}
          \item Hohe Frequenzen werden stark quantisiert (durch große Werte in $Q$), oft zu Null.
          \item Steuert die Qualität vs. Kompressionsrate.
        \end{itemize}

  \item \textbf{Kodierung (Entropy Coding):}
        \begin{itemize}
          \item \defc{Zig-Zag-Scan}: Sortiert die Koeffizienten von niedrigen zu hohen Frequenzen, um lange Ketten von Nullen zu erzeugen.
          \item \textbf{RLE (Run-Length Encoding):} Kodiert die Nullen-Ketten effizient.
          \item \textbf{Huffman-Kodierung:} Weist häufigen Werten kurze Bitfolgen zu.
        \end{itemize}
\end{enumerate}

\end{document}