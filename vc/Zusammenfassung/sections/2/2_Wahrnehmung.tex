% !TEX root = ./vc_summary.tex

\section{Wahrnehmung (Perception)}
\label{sec:wahrnehmung}

\subsection{Motivation und Kognition (Motivation and Cognition)}
\label{ssec:motivation_kognition}
\begin{itemize}
    \item \textbf{Warum VC? (Why VC?)} Die Leistungsf\"ahigkeit von Rechnern w\"achst exponentiell (\textbf{Moore's Law}), aber die Kapazit\"at von Menschen ist (fast) konstant (\textbf{Darwin's Law}). VC hilft, diese L\"ucke zu \"uberbr\"ucken.
    \item \textbf{Kognition (Cognition):} Sammelbegriff f\"ur alle Prozesse des Wahrnehmens und Erkennens (Denken, Erinnern, Lernen, etc.).
    \item \textbf{Modell der Informationsverarbeitung (Model of Information Processing):} Ein modulares 3-Stufenmodell:
    \begin{enumerate}
        \item \textbf{Perception} (Wahrnehmung durch Sinne)
        \item \textbf{Decision} (Entscheidungsfindung im Gehirn)
        \item \textbf{Response} (Reaktion durch K\"orper)
    \end{enumerate}
    \item \textbf{Bearbeitungszeiten (Processing Times):} Jedes Untersystem ben\"otigt Zeit.
    \begin{itemize}
        \item Wahrnehmung (Perception): $\approx$ 100 ms
        \item Entscheidung (Cognition): $\approx$ 70 ms
        \item Reaktion (Motor): $\approx$ 70 ms
    \end{itemize}
    \item \textbf{Wahrnehmung vs. Realit\"at:} Was wir wahrnehmen, ist kein direktes Abbild der Realit\"at, sondern eine partielle Hypothese, die auf unvollst\"andiger Information basiert.
\end{itemize}

\subsection{Das Visuelle System (The Visual System)}
\label{ssec:visual_system}

\begin{defbox}[Visueller Reiz (Visual Stimulus)]
    Ein \"au\ss erer visueller Reiz ist \textbf{elektromagnetische Strahlung}. Sichtbares Licht liegt im Wellenl\"angenbereich von ca. 400nm (violett) bis 700nm (rot).
    Die Frequenz $v$ und Wellenl\"ange $\lambda$ h\"angen \"uber $v \cdot \lambda = c$ zusammen.
\end{defbox}

\subsubsection{Aufbau des Auges (Structure of the Eye)}
\begin{center}
  \includegraphics[width=250px]{pics/vc_1_VC2025_02_Wahrnehmung_page_30_1.png}
\end{center}
\begin{itemize}
    \item \textbf{Optische Elemente:} Hornhaut (Kornea), Linse, Iris (Blende, 2-8mm), Glask\"orper.
    \item \textbf{Linse (Lens):} Akkomodation (Scharfeinstellung).
    \item \textbf{Retina (Netzhaut):} Enth\"alt die Photorezeptoren.
    \item \textbf{Fovea Centralis:} Bereich der h\"ochsten Aufl\"osung (im "gelben Fleck" / Macula lutea).
    \item \textbf{Blinder Fleck (Blind Spot):} Papilla nervi optici; Austrittspunkt des Sehnervs, keine Rezeptoren.
\end{itemize}


\subsubsection{Photorezeptoren (Photoreceptors)}
Es gibt zwei Haupttypen von Photorezeptoren auf der Retina:
\begin{itemize}
    \item \textbf{St\"abchen (Rods):}
    \begin{itemize}
        \item ca. 100-120 Mio.
        \item Haupts\"achlich au\ss erhalb der Fovea.
        \item F\"ur \textbf{Nachtsehen (skotopisches Sehen)}.
        \item Sehr lichtempfindlich, kein Farbsehen.
        \item Empfindlichkeitsmaximum bei 498 nm (gr\"un).
    \end{itemize}
    \item \textbf{Zapfen (Cones):}
    \begin{itemize}
        \item ca. 7-8 Mio.
        \item Haupts\"achlich \textbf{in der Fovea} (Bereich des sch\"arfsten Sehens).
        \item F\"ur \textbf{Tagsehen (photopisches Sehen)}.
        \item 3 Typen f\"ur Farbsehen:
        \begin{itemize}
            \item \textbf{S-Zapfen} (Short): Max. bei 420 nm (Blau).
            \item \textbf{M-Zapfen} (Medium): Max. bei 534 nm (Gr\"un).
            \item \textbf{L-Zapfen} (Long): Max. bei 564 nm (Rot).
        \end{itemize}
    \end{itemize}
\end{itemize}
\begin{center}
  \includegraphics[width=250px]{pics/vc_1_VC2025_02_Wahrnehmung_page_42_1.png}
\end{center}

\begin{defbox}[Bayer-Sensor]
    Digitale Kamerasensoren nutzen oft ein \textbf{Bayer-Muster}. Dies ist ein Farbfilter-Array, meist mit \textbf{50\% Gr\"un, 25\% Rot und 25\% Blau}. Gr\"un ist privilegiert, da das menschliche Auge f\"ur Gr\"un den gr\"o\ss ten Beitrag zur \textbf{Helligkeits- und Kontrastwahrnehmung} leistet (72\% Gr\"unanteil).
    \begin{center}
  \includegraphics[width=150px]{pics/vc_1_VC2025_02_Wahrnehmung_page_39_1.png}
\end{center}
\end{defbox}

\subsection{Vorverarbeitung \& Helligkeit (Preprocessing \& Brightness)}
\label{ssec:preprocessing_brightness}

\subsubsection{Signalverarbeitung in der Retina}
Das Licht trifft (paradoxerweise) erst auf die Ganglien- und Bipolarzellen, bevor es die St\"abchen und Zapfen erreicht.
\begin{itemize}
    \item \textbf{Bipolar-Zellen:} Sammeln, gewichten und leiten Informationen weiter.
    \item \textbf{Horizontal- \& Amakrin-Zellen:} Kombinieren Signale mehrerer Rezeptoren (r\"aumlich) bzw. verarbeiten zeitliche \"Anderungen.
    \item \textbf{Ganglien-Zellen:} Integrieren Informationen, z.B. f\"ur \textbf{Kontrastwahrnehmung} durch Unterschied zwischen Zentrum und Peripherie (Center-Surround-Antagonismus).
\end{itemize}
\begin{center}
  \includegraphics[width=200px]{pics/vc_1_VC2025_02_Wahrnehmung_page_47_2.png}
  \includegraphics[width=200px]{pics/vc_1_VC2025_02_Wahrnehmung_page_48_1.png}
\end{center}


\subsubsection{Optische T\"auschungen (Optical Illusions)}
Diese fr\"uhe Signalverarbeitung f\"uhrt zu T\"auschungen, die zeigen, dass Wahrnehmung nicht objektiv ist:
\begin{itemize}
    \item \textbf{Hermann-Gitter:} Graue Flecke erscheinen in den Kreuzungen eines wei\ss en Gitters auf schwarzem Grund.
    \item \textbf{Mach-B\"ander (Mach Bands):} An Kanten zwischen unterschiedlich hellen, aber homogenen Fl\"achen werden helle/dunkle B\"ander wahrgenommen, wo keine sind (eine Art "Uberschwingen" der Wahrnehmung).
    \item \textbf{Simultankontrast (Simultaneous Contrast):} Die wahrgenommene Helligkeit einer Fl\"ache h\"angt von der Helligkeit ihrer Umgebung ab. Ein identisches Grau erscheint auf schwarzem Grund heller als auf wei\ss em Grund.
\end{itemize}

\subsubsection{Helligkeitswahrnehmung (Brightness Perception)}
\begin{itemize}
    \item Helligkeit (Brightness) ist \textbf{keine absolute Gr\"o\ss e}, sondern subjektiv.
    \item Sie ist u.a. abh\"angig von der Reizst\"arke (Leuchtdichte), der Adaption an vorherige Leuchtdichten und der Umgebungsleuchtdichte.
    \item \textbf{Weber-Fechnersches Gesetz:} Beschreibt den Zusammenhang zwischen Reizintensit\"at (R) und Hellempfindung (L).
    \begin{itemize}
        \item \textbf{Webersches Gesetz (Schwelle):} $\Delta L = \frac{\Delta R}{R} = const.$ (minimaler Kontrast f\"ur Wahrnehmung ca. 0.8\%)
        \item \textbf{Fechnersches Gesetz:} $L = c_{1} \times \log R$
        \item \textbf{Stevensches Gesetz (State-of-the-Art):} $E = c_{2} \times R^{k}$ (f\"ur Licht $k=0.3$)
    \end{itemize}
\end{itemize}

\subsubsection{Aufl\"osung und Kontrast (Resolution and Contrast)}
\begin{itemize}
    \item \textbf{Sehsch\"arfe (Visual Acuity):} Die F\"ahigkeit, kleine Details zu erkennen, ist begrenzt. Z.B. Punktsehsch\"arfe ca. 1 Bogenminute ($1'$).
    \item \textbf{Kontrastempfindlichkeit (Contrast Sensitivity):} Gemessen mit Sinus-Mustern (sinusoidal gratings).
    \item \textbf{Contrast Sensitivity Function (CSF):} Beschreibt die Aufl\"osung im Frequenzraum. Zeigt, dass das Auge f\"ur mittlere Ortsfrequenzen (ca. 2-5 Zyklen/Grad) am empfindlichsten ist und die Empfindlichkeit zu sehr hohen (Details) und sehr niedrigen Frequenzen (langsame \"Uberg\"ange) abf\"allt.
\end{itemize}

\subsection{Informationsextraktion: Tiefenwahrnehmung (Depth Perception)}
\label{ssec:depth_perception}
Das visuelle System nutzt verschiedene Hinweisreize (\textbf{Depth Cues}), um Raumwahrnehmung zu erzeugen.

\begin{itemize}
    \item \textbf{1. Binokulare Cues (Zwei Augen):}
    \begin{itemize}
        \item \textbf{Disparit\"at / Parallaxe:} Der Haupt-Cue. Da die Augen getrennt sind, sehen sie leicht unterschiedliche Bilder. Das Gehirn fusioniert diese.
        \item \textbf{Positive Parallaxe:} Objekte erscheinen \textit{hinter} der Bildebene.
        \item \textbf{Negative Parallaxe:} Objekte erscheinen \textit{vor} der Bildebene.
        \item \textbf{Akkomodation} (Linsenanpassung) und \textbf{Konvergenz} (Augenstellung).
    \end{itemize}
    
    \item \textbf{2. Pictorial Depth Cues (Monokular / Bildlich):}
    \begin{itemize}
        \item \textbf{Linearperspektive:} Parallele Linien konvergieren in der Ferne.
        \item \textbf{Verdeckung (Occlusion):} Ein Objekt, das ein anderes verdeckt, wird als n\"aher wahrgenommen.
        \item \textbf{Texturgradient:} Texturen werden mit der Entfernung dichter und feiner.
        \item \textbf{Atmosph\"arische Tiefe:} Entfernte Objekte erscheinen blasser und bl\"aulicher.
        \item \textbf{Schattenwurf (Shadows):} Wichtig f\"ur Position und Form; Annahme: Licht kommt von oben.
        \item Weitere: Fokus/Blur, Vertraute Gr\"o\ss e, H\"ohe im Gesichtsfeld, etc..
    \end{itemize}
    
    \item \textbf{3. Dynamische Depth Cues (Bewegung):}
    \begin{itemize}
        \item \textbf{Bewegungsparallaxe (Motion Parallax):} Objekte, die n\"aher sind, bewegen sich bei einer Kopfbewegung scheinbar schneller als entfernte Objekte.
        \item \textbf{Kinetischer Tiefeneffekt (Kinetic Depth Effect):} 3D-Struktur wird aus der 2D-Projektion einer Bewegung extrahiert (z.B. "Structure from Motion").
    \end{itemize}
\end{itemize}
Die Depth Cues sind nicht redundant, sondern \textbf{additiv} und werden je nach Aufgabe (Task) \textbf{flexibel gewichtet}.

\subsection{Aufmerksamkeit und Ged\"achtnis (Attention and Memory)}
\label{ssec:attention_memory}
\begin{itemize}
    \item \textbf{Fr\"uhe Wahrnehmung (Preattentive Processing):} Bestimmte Merkmale (Farbe, Gr\"o\ss e, Richtung, Schattierung) werden sehr schnell (ca. < 10 ms) und parallel verarbeitet, bevor die bewusste Aufmerksamkeit greift. \textbf{Verbindungen} von Merkmalen (z.B. "roter Kreis") erfordern Aufmerksamkeit.
    \item \textbf{Aufmerksamkeit (Attention):} Dient als \textbf{Filter} oder "Gateway to Memory".
    \item \textbf{Ver\"anderungsblindheit (Change Blindness):} Unf\"ahigkeit, gro\ss e \"Anderungen in einer Szene zu bemerken, wenn die Aufmerksamkeit abgelenkt ist (z.B. durch Flimmern). Dies zeigt, dass wir kein vollst\"andiges Bild der Welt im Kopf haben.
    \item \textbf{Arbeitsged\"achtnis (Working Memory):}
    \begin{itemize}
        \item Schneller Zugriff (ca. 70 ms), schneller Verfall (ca. 200 ms).
        \item Sehr begrenzte Kapazit\"at: \textbf{$7 \pm 2$ "Chunks"} (Miller, 1956).
        \item "Chunks" sind sinnvolle Einheiten (z.B. DA, TU, VC, VL statt DATUVCVL).
    \end{itemize}
    \item \textbf{Langzeitged\"achtnis (Long-term Memory):}
    \begin{itemize}
        \item Nahezu unbegrenzte Kapazit\"at.
        \item Langsamerer Zugriff (ca. 100 ms).
    \end{itemize}
\end{itemize}