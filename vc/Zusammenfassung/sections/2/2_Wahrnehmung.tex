\documentclass[
../../vc_summary.tex,
]
{subfiles}

\externaldocument[ext:]{../../vc_summary}
% Set Graphics Path, so pictures load correctly
\graphicspath{{../../pics}}

\begin{document}

\section{Wahrnehmung (Perception)}
\label{sec:wahrnehmung}

\subsection{Motivation und Kognition (Motivation and Cognition)}
\label{ssec:motivation_kognition}
\begin{itemize}
   \item \textbf{Warum VC? (Why VC?)} Die Leistungsfähigkeit von Rechnern wächst exponentiell (\textbf{Moore's Law}), aber die Kapazität von Menschen ist (fast) konstant (\textbf{Darwin's Law}). VC hilft, diese Lücke zu überbrücken.
   \item \textbf{Kognition (Cognition):} Sammelbegriff für alle Prozesse des Wahrnehmens und Erkennens (Denken, Erinnern, Lernen, etc.).
   \item \textbf{Modell der Informationsverarbeitung (Model of Information Processing):} Ein modulares 3-Stufenmodell:
         \begin{enumerate}
            \item \textbf{Perception} (Wahrnehmung durch Sinne)
            \item \textbf{Decision} (Entscheidungsfindung im Gehirn)
            \item \textbf{Response} (Reaktion durch Körper)
         \end{enumerate}
   \item \textbf{Bearbeitungszeiten (Processing Times):} Jedes Untersystem benötigt Zeit.
         \begin{itemize}
            \item Wahrnehmung (Perception): $\approx$ 100 ms
            \item Entscheidung (Cognition): $\approx$ 70 ms
            \item Reaktion (Motor): $\approx$ 70 ms
         \end{itemize}
   \item \textbf{Wahrnehmung vs. Realität:} Was wir wahrnehmen, ist kein direktes Abbild der Realität, sondern eine partielle Hypothese, die auf unvollständiger Information basiert.
\end{itemize}

\subsection{Das Visuelle System (The Visual System)}
\label{ssec:visual_system}

\begin{defbox}[Visueller Reiz (Visual Stimulus)]
   Ein äu\ss erer visueller Reiz ist \textbf{elektromagnetische Strahlung}. Sichtbares Licht liegt im Wellenlängenbereich von ca. 400nm (violett) bis 700nm (rot).
   Die Frequenz $v$ und Wellenlänge $\lambda$ hängen über $v \cdot \lambda = c$ zusammen.
\end{defbox}

\subsubsection{Aufbau des Auges (Structure of the Eye)}
\begin{center}
   \includegraphics[width=250px]{vc_1_VC2025_02_Wahrnehmung_page_30_1.png}
\end{center}
\begin{itemize}
   \item \textbf{Optische Elemente:} Hornhaut (Kornea), Linse, Iris (Blende, 2-8mm), Glaskörper.
   \item \textbf{Linse (Lens):} Akkomodation (Scharfeinstellung).
   \item \textbf{Retina (Netzhaut):} Enthält die Photorezeptoren.
   \item \textbf{Fovea Centralis:} Bereich der höchsten Auflösung (im ``gelben Fleck'' / Macula lutea).
   \item \textbf{Blinder Fleck (Blind Spot):} Papilla nervi optici; Austrittspunkt des Sehnervs, keine Rezeptoren.
\end{itemize}


\subsubsection{Photorezeptoren (Photoreceptors)}
Es gibt zwei Haupttypen von Photorezeptoren auf der Retina:
\begin{itemize}
   \item \textbf{Stäbchen (Rods):}
         \begin{itemize}
            \item ca. 100-120 Mio.
            \item Hauptsächlich au\ss erhalb der Fovea.
            \item Für \textbf{Nachtsehen (skotopisches Sehen)}.
            \item Sehr lichtempfindlich, kein Farbsehen.
            \item Empfindlichkeitsmaximum bei 498 nm (grün).
         \end{itemize}
   \item \textbf{Zapfen (Cones):}
         \begin{itemize}
            \item ca. 7-8 Mio.
            \item Hauptsächlich \textbf{in der Fovea} (Bereich des schärfsten Sehens).
            \item Für \textbf{Tagsehen (photopisches Sehen)}.
            \item 3 Typen für Farbsehen:
                  \begin{itemize}
                     \item \textbf{S-Zapfen} (Short): Max. bei 420 nm (Blau).
                     \item \textbf{M-Zapfen} (Medium): Max. bei 534 nm (Grün).
                     \item \textbf{L-Zapfen} (Long): Max. bei 564 nm (Rot).
                  \end{itemize}
         \end{itemize}
\end{itemize}
\begin{center}
   \includegraphics[width=250px]{vc_1_VC2025_02_Wahrnehmung_page_42_1.png}
\end{center}

\begin{defbox}[Bayer-Sensor]
   Digitale Kamerasensoren nutzen oft ein \textbf{Bayer-Muster}. Dies ist ein Farbfilter-Array, meist mit \textbf{50\% Grün, 25\% Rot und 25\% Blau}. Grün ist privilegiert, da das menschliche Auge für Grün den grö\ss ten Beitrag zur \textbf{Helligkeits- und Kontrastwahrnehmung} leistet (72\% Grünanteil).
   \begin{center}
      \includegraphics[width=150px]{vc_1_VC2025_02_Wahrnehmung_page_39_1.png}
   \end{center}
\end{defbox}

\subsection{Vorverarbeitung \& Helligkeit (Preprocessing \& Brightness)}
\label{ssec:preprocessing_brightness}

\subsubsection{Signalverarbeitung in der Retina}
Das Licht trifft (paradoxerweise) erst auf die Ganglien- und Bipolarzellen, bevor es die Stäbchen und Zapfen erreicht.
\begin{itemize}
   \item \textbf{Bipolar-Zellen:} Sammeln, gewichten und leiten Informationen weiter.
   \item \textbf{Horizontal- \& Amakrin-Zellen:} Kombinieren Signale mehrerer Rezeptoren (räumlich) bzw. verarbeiten zeitliche änderungen.
   \item \textbf{Ganglien-Zellen:} Integrieren Informationen, z.B. für \textbf{Kontrastwahrnehmung} durch Unterschied zwischen Zentrum und Peripherie (Center-Surround-Antagonismus).
\end{itemize}
\begin{center}
   \includegraphics[width=200px]{vc_1_VC2025_02_Wahrnehmung_page_47_2.png}
   \includegraphics[width=200px]{vc_1_VC2025_02_Wahrnehmung_page_48_1.png}
\end{center}


\subsubsection{Optische Täuschungen (Optical Illusions)}
Diese frühe Signalverarbeitung führt zu Täuschungen, die zeigen, dass Wahrnehmung nicht objektiv ist:
\begin{itemize}
   \item \textbf{Hermann-Gitter:} Graue Flecke erscheinen in den Kreuzungen eines wei\ss en Gitters auf schwarzem Grund.
   \item \textbf{Mach-Bänder (Mach Bands):} An Kanten zwischen unterschiedlich hellen, aber homogenen Flächen werden helle/dunkle Bänder wahrgenommen, wo keine sind (eine Art ``Uberschwingen'' der Wahrnehmung).
   \item \textbf{Simultankontrast (Simultaneous Contrast):} Die wahrgenommene Helligkeit einer Fläche hängt von der Helligkeit ihrer Umgebung ab. Ein identisches Grau erscheint auf schwarzem Grund heller als auf wei\ss em Grund.
\end{itemize}

\subsubsection{Helligkeitswahrnehmung (Brightness Perception)}
\begin{itemize}
   \item Helligkeit (Brightness) ist \textbf{keine absolute Grö\ss e}, sondern subjektiv.
   \item Sie ist u.a. abhängig von der Reizstärke (Leuchtdichte), der Adaption an vorherige Leuchtdichten und der Umgebungsleuchtdichte.
   \item \textbf{Weber-Fechnersches Gesetz:} Beschreibt den Zusammenhang zwischen Reizintensität (R) und Hellempfindung (L).
         \begin{itemize}
            \item \textbf{Webersches Gesetz (Schwelle):} $\Delta L = \frac{\Delta R}{R} = const.$ (minimaler Kontrast für Wahrnehmung ca. 0.8\%)
            \item \textbf{Fechnersches Gesetz:} $L = c_{1} \times \log R$
            \item \textbf{Stevensches Gesetz (State-of-the-Art):} $E = c_{2} \times R^{k}$ (für Licht $k=0.3$)
         \end{itemize}
\end{itemize}

\subsubsection{Auflösung und Kontrast (Resolution and Contrast)}
\begin{itemize}
   \item \textbf{Sehschärfe (Visual Acuity):} Die Fähigkeit, kleine Details zu erkennen, ist begrenzt. Z.B. Punktsehschärfe ca. 1 Bogenminute ($1'$).
   \item \textbf{Kontrastempfindlichkeit (Contrast Sensitivity):} Gemessen mit Sinus-Mustern (sinusoidal gratings).
   \item \textbf{Contrast Sensitivity Function (CSF):} Beschreibt die Auflösung im Frequenzraum. Zeigt, dass das Auge für mittlere Ortsfrequenzen (ca. 2-5 Zyklen/Grad) am empfindlichsten ist und die Empfindlichkeit zu sehr hohen (Details) und sehr niedrigen Frequenzen (langsame Übergänge) abfällt.
\end{itemize}

\subsection{Informationsextraktion: Tiefenwahrnehmung (Depth Perception)}
\label{ssec:depth_perception}
Das visuelle System nutzt verschiedene Hinweisreize (\textbf{Depth Cues}), um Raumwahrnehmung zu erzeugen.

\begin{itemize}
   \item \textbf{1. Binokulare Cues (Zwei Augen):}
         \begin{itemize}
            \item \textbf{Disparität / Parallaxe:} Der Haupt-Cue. Da die Augen getrennt sind, sehen sie leicht unterschiedliche Bilder. Das Gehirn fusioniert diese.
            \item \textbf{Positive Parallaxe:} Objekte erscheinen \textit{hinter} der Bildebene.
            \item \textbf{Negative Parallaxe:} Objekte erscheinen \textit{vor} der Bildebene.
            \item \textbf{Akkomodation} (Linsenanpassung) und \textbf{Konvergenz} (Augenstellung).
         \end{itemize}

   \item \textbf{2. Pictorial Depth Cues (Monokular / Bildlich):}
         \begin{itemize}
            \item \textbf{Linearperspektive:} Parallele Linien konvergieren in der Ferne.
            \item \textbf{Verdeckung (Occlusion):} Ein Objekt, das ein anderes verdeckt, wird als näher wahrgenommen.
            \item \textbf{Texturgradient:} Texturen werden mit der Entfernung dichter und feiner.
            \item \textbf{Atmosphärische Tiefe:} Entfernte Objekte erscheinen blasser und bläulicher.
            \item \textbf{Schattenwurf (Shadows):} Wichtig für Position und Form; Annahme: Licht kommt von oben.
            \item Weitere: Fokus/Blur, Vertraute Grö\ss e, Höhe im Gesichtsfeld, etc..
         \end{itemize}

   \item \textbf{3. Dynamische Depth Cues (Bewegung):}
         \begin{itemize}
            \item \textbf{Bewegungsparallaxe (Motion Parallax):} Objekte, die näher sind, bewegen sich bei einer Kopfbewegung scheinbar schneller als entfernte Objekte.
            \item \textbf{Kinetischer Tiefeneffekt (Kinetic Depth Effect):} 3D-Struktur wird aus der 2D-Projektion einer Bewegung extrahiert (z.B. ``Structure from Motion'').
         \end{itemize}
\end{itemize}
Die Depth Cues sind nicht redundant, sondern \textbf{additiv} und werden je nach Aufgabe (Task) \textbf{flexibel gewichtet}.

\subsection{Aufmerksamkeit und Gedächtnis (Attention and Memory)}
\label{ssec:attention_memory}
\begin{itemize}
   \item \textbf{Frühe Wahrnehmung (Preattentive Processing):} Bestimmte Merkmale (Farbe, Grö\ss e, Richtung, Schattierung) werden sehr schnell (ca. < 10 ms) und parallel verarbeitet, bevor die bewusste Aufmerksamkeit greift. \textbf{Verbindungen} von Merkmalen (z.B. ``roter Kreis'') erfordern Aufmerksamkeit.
   \item \textbf{Aufmerksamkeit (Attention):} Dient als \textbf{Filter} oder ``Gateway to Memory''.
   \item \textbf{Veränderungsblindheit (Change Blindness):} Unfähigkeit, gro\ss e änderungen in einer Szene zu bemerken, wenn die Aufmerksamkeit abgelenkt ist (z.B. durch Flimmern). Dies zeigt, dass wir kein vollständiges Bild der Welt im Kopf haben.
   \item \textbf{Arbeitsgedächtnis (Working Memory):}
         \begin{itemize}
            \item Schneller Zugriff (ca. 70 ms), schneller Verfall (ca. 200 ms).
            \item Sehr begrenzte Kapazität: \textbf{$7 \pm 2$ ``Chunks''} (Miller, 1956).
            \item ``Chunks'' sind sinnvolle Einheiten (z.B. DA, TU, VC, VL statt DATUVCVL).
         \end{itemize}
   \item \textbf{Langzeitgedächtnis (Long-term Memory):}
         \begin{itemize}
            \item Nahezu unbegrenzte Kapazität.
            \item Langsamerer Zugriff (ca. 100 ms).
         \end{itemize}
\end{itemize}

\end{document}