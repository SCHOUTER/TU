\documentclass[11pt,a4paper]{article}
\title{\"Ubung 6}
\author{Niclas Kusenbach, 360227}

\usepackage[T1]{fontenc}
\usepackage[utf8]{inputenc}
\usepackage[ngerman,british]{babel}
\usepackage{amsmath,amsfonts,amssymb}
\usepackage{booktabs}
\usepackage{graphicx}
\usepackage{array}
\usepackage{siunitx}
\usepackage{physics}
\usepackage{xcolor}
\usepackage{enumitem}
\usepackage[final]{microtype} %Fix most of overfull hboxes
\usepackage{xurl} % better line breaks for \url and plain URLs
\usepackage{hyperref} % better line breaks for \url and plain URLs
\usepackage{tabularx}
\setlist{nosep}

\begin{document}

\maketitle

\section*{Task 1: Propositional Logic}

\subsection*{1a) Logische Äquivalenzen}
\textit{Beweisen Sie die Aussagen durch Umformung.}

\begin{enumerate}[label=\alph*)]
  \item $P \Rightarrow Q \equiv \neg(P \wedge \neg Q)$
        \begin{align*}
          \text{Rechte Seite: } & \neg(P \wedge \neg Q)                                            \\
          \equiv \quad          & \neg P \vee \neg(\neg Q) \quad \text{(De Morgan)}                \\
          \equiv \quad          & \neg P \vee Q \quad \text{(Doppelte Negation)}                   \\
          \equiv \quad          & P \Rightarrow Q \quad \text{(Def. Implikation)} \quad \checkmark
        \end{align*}

  \item $(P \wedge Q) \Rightarrow R \equiv P \Rightarrow (Q \Rightarrow R)$
        \begin{align*}
          \text{Linke Seite: }  & (P \wedge Q) \Rightarrow R                         \\
          \equiv \quad          & \neg(P \wedge Q) \vee R                            \\
          \equiv \quad          & \neg P \vee \neg Q \vee R \quad \text{(De Morgan)} \\[1em]
          \text{Rechte Seite: } & P \Rightarrow (Q \Rightarrow R)                    \\
          \equiv \quad          & \neg P \vee (\neg Q \vee R)                        \\
          \equiv \quad          & \neg P \vee \neg Q \vee R \quad \checkmark
        \end{align*}
\end{enumerate}

\subsection*{1b) Logische Folgerung (Wahrheitstabellen)}
\textit{Prüfen Sie die Folgerungen ($\models$).}

\begin{enumerate}[label=\alph*)]
  \item $P \wedge Q \models P \vee Q$ \\
        \textbf{Gilt.} Wenn $P \wedge Q$ wahr ist ($P=1, Q=1$), dann ist $1 \vee 1 = 1$ (wahr).

  \item $P \wedge \neg P \models Q$ \\
        \textbf{Gilt.} Die Prämisse $P \wedge \neg P$ ist ein Widerspruch (immer falsch). Aus Falschem folgt Beliebiges (\textit{ex falso quodlibet}).

  \item $P \Rightarrow (P \wedge Q) \models P \vee \neg Q$ \\
        \textbf{Gilt NICHT.} Gegenbeispiel: $P=0, Q=1$.
        \begin{itemize}
          \item Prämisse: $0 \Rightarrow (0 \wedge 1) \equiv \text{Wahr}$.
          \item Konklusion: $0 \vee \neg 1 \equiv 0 \vee 0 \equiv \text{Falsch}$.
        \end{itemize}
\end{enumerate}

\subsection*{1c) Konjunktive Normalform (KNF)}
\textit{Umwandlung in KNF.}

\begin{enumerate}[label=\alph*)]
  \item $(P \wedge Q) \Rightarrow (\neg P \Leftrightarrow Q)$
        \begin{align*}
                       & \neg(P \wedge Q) \vee ((\neg P \wedge Q) \vee (P \wedge \neg Q))   \\
          \equiv \quad & (\neg P \vee \neg Q) \vee (\neg P \wedge Q) \vee (P \wedge \neg Q) \\
          \equiv \quad & \neg P \vee \neg Q \quad \text{(Absorption)}
        \end{align*}

  \item $((P \Rightarrow Q) \wedge \neg Q) \Rightarrow \neg P$
        \begin{align*}
                       & \neg ((\neg P \vee Q) \wedge \neg Q) \vee \neg P                   \\
          \equiv \quad & \neg (\neg P \wedge \neg Q) \vee \neg P                            \\
          \equiv \quad & (P \vee Q) \vee \neg P \quad \equiv \quad \text{True (Tautologie)}
        \end{align*}

  \item $(( \neg P \wedge Q ) \vee (\neg R \wedge S)) \wedge (U \vee V)$
        \begin{align*}
          \text{Distributivgesetz: } & (A \wedge B) \vee (C \wedge D) \equiv (A \vee C) \wedge (A \vee D) \wedge (B \vee C) \wedge (B \vee D) \\
          \text{Ergebnis: }          & (\neg P \vee \neg R) \wedge (\neg P \vee S) \wedge (Q \vee \neg R) \wedge (Q \vee S) \wedge (U \vee V)
        \end{align*}

  \item $((\neg P \vee Q) \wedge (\neg R \vee S)) \vee (U \wedge V)$
        \begin{align*}
          \text{Ergebnis: } & (\neg P \vee Q \vee U) \wedge (\neg P \vee Q \vee V) \wedge (\neg R \vee S \vee U) \wedge (\neg R \vee S \vee V)
        \end{align*}
\end{enumerate}

\newpage

\section*{Task 2: Resolution}
\textit{Beweis durch Widerspruch: Füge $\neg \psi$ zur KB hinzu und leite die leere Klausel ($\square$) ab.}

\subsection*{2a)}
\textbf{Gegeben:} KB $= \{P \wedge Q\}$, $\psi = (P \vee Q)$ \\
\textbf{Negiertes Ziel:} $\neg(P \vee Q) \equiv \neg P \wedge \neg Q$

\textbf{Klauselmenge:}
\begin{enumerate}
  \item $P$
  \item $Q$
  \item $\neg P$ (aus negiertem Ziel)
  \item $\neg Q$ (aus negiertem Ziel)
\end{enumerate}

\textbf{Resolution:}
\begin{align*}
  (1) \text{ und } (3) \quad \rightarrow \quad \square \quad \text{(Widerspruch)}
\end{align*}

\subsection*{2b)}
\textbf{Gegeben:} KB $= \{P \vee Q, Q \Rightarrow (R \wedge S), (P \vee R) \Rightarrow U\}$, $\psi = U$ \\
\textbf{Negiertes Ziel:} $\neg U$

\textbf{Umwandlung in Klauseln:}
\begin{itemize}
  \item $Q \Rightarrow (R \wedge S) \equiv \neg Q \vee (R \wedge S) \rightarrow \{ \neg Q \vee R, \quad \neg Q \vee S \}$
  \item $(P \vee R) \Rightarrow U \equiv (\neg P \wedge \neg R) \vee U \rightarrow \{ \neg P \vee U, \quad \neg R \vee U \}$
\end{itemize}

\textbf{Klauselmenge:}
\begin{enumerate}
  \item $P \vee Q$
  \item $\neg Q \vee R$
  \item $\neg Q \vee S$
  \item $\neg P \vee U$
  \item $\neg R \vee U$
  \item $\neg U$ \quad (Negiertes Ziel)
\end{enumerate}

\textbf{Resolutionsbeweis:}
\begin{align*}
  1. \quad & \text{Res. (6) } \neg U \text{ mit (5) } \neg R \vee U & \rightarrow \quad & \mathbf{\neg R}  \\
  2. \quad & \text{Res. (6) } \neg U \text{ mit (4) } \neg P \vee U & \rightarrow \quad & \mathbf{\neg P}  \\
  3. \quad & \text{Res. (neu) } \neg P \text{ mit (1) } P \vee Q    & \rightarrow \quad & \mathbf{Q}       \\
  4. \quad & \text{Res. (neu) } Q \text{ mit (2) } \neg Q \vee R    & \rightarrow \quad & \mathbf{R}       \\
  5. \quad & \text{Res. (neu) } R \text{ mit (neu) } \neg R         & \rightarrow \quad & \mathbf{\square}
\end{align*}

\section*{Task 3: General Questions}

\subsection*{3a) Probleme und Grenzen der Aussagenlogik}
Haupteinschränkungen bei der Anwendung auf reale Probleme:

\begin{itemize}
  \item \textbf{Mangelnde Ausdrucksstärke:} Objekte, Relationen und Quantoren ("Alle Menschen sind sterblich") können nicht direkt modelliert werden. Jedes Faktum muss einzeln atomar definiert werden.
  \item \textbf{Zustandsraum-Explosion:} Die Größe der Wahrheitstabellen wächst exponentiell ($2^n$). Bei vielen Variablen wird Inferenz unberechenbar (NP-vollständig).
  \item \textbf{Statische Natur:} Zeitliche Veränderungen (Temporal) oder Unsicherheiten (Probabilistik) sind in der klassischen Aussagenlogik nicht abbildbar.
  \item \textbf{Monotonie:} Annahmen können nicht zurückgenommen werden, wenn neue Informationen hinzukommen (kein "non-monotonic reasoning").
\end{itemize}

\end{document}