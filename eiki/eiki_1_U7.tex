\documentclass[11pt,a4paper]{article}
\title{\"Ubung 7}
\author{Niclas Kusenbach, 360227}

\usepackage[T1]{fontenc}
\usepackage[utf8]{inputenc}
\usepackage[ngerman,british]{babel}
\usepackage{amsmath,amsfonts,amssymb}
\usepackage{booktabs}
\usepackage{graphicx}
\usepackage{array}
\usepackage{siunitx}
\usepackage{physics}
\usepackage{xcolor}
\usepackage{enumitem}
\usepackage[final]{microtype} %Fix most of overfull hboxes
\usepackage{xurl} % better line breaks for \url and plain URLs
\usepackage{hyperref} % better line breaks for \url and plain URLs
\usepackage{tabularx}
\setlist{nosep}

\begin{document}

\maketitle

\section*{Aufgabe 1: Logik erster Stufe}

\subsection*{1a) Es gibt mindestens zwei Berge.}
\textbf{Prädikat:} $Berg(x)$: $x$ ist ein Berg. \\
Um „mindestens zwei“ auszudrücken, müssen zwei existieren, die nicht identisch sind.

$$ \exists x \exists y (Berg(x) \land Berg(y) \land \neg(x = y)) $$

\subsection*{1b) Es liegt genau eine Münze in der Kiste.}
\textbf{Prädikate:} $Muenze(x)$: $x$ ist eine Münze, $InKiste(x)$: $x$ liegt in der Kiste. \\
„Genau eine“ bedeutet: Es gibt eine (Existenz) und für alle anderen, die diese Eigenschaft haben, gilt, dass sie mit der ersten identisch sind (Einzigkeit).

$$ \exists x (Muenze(x) \land InKiste(x) \land \forall y ((Muenze(y) \land InKiste(y)) \rightarrow x = y)) $$

\subsection*{1c) Ein Rechtsgeschäft, das gegen die guten Sitten verstößt, ist nichtig.}
\textbf{Prädikate:} $RG(x)$: $x$ ist Rechtsgeschäft, $Sittenwidrig(x)$: $x$ verstößt gegen Sitten, $Nichtig(x)$: $x$ ist nichtig. \\
Dies ist eine All-Aussage.

$$ \forall x ((RG(x) \land Sittenwidrig(x)) \rightarrow Nichtig(x)) $$

\hrulefill

\section*{Aufgabe 2: Resolution in FOL}

\subsection*{2a) Formalisierung in FOL}
\textbf{Prädikate:} $Hat(x,y)$, $Hund(x)$, $Heult(x)$, $Katze(x)$, $Maus(x)$, $LS(x)$ (Leichtschläfer). \\
\textbf{Konstante:} $John$.

\begin{enumerate}
      \item[\textbf{i)}] Alle Hunde heulen in der Nacht.
            $$ \forall x (Hund(x) \rightarrow Heult(x)) $$
      \item[\textbf{ii)}] Jeder der Katzen hat, hat keine Mäuse.
            $$ \forall x ((\exists y (Katze(y) \land Hat(x, y))) \rightarrow \neg \exists z (Maus(z) \land Hat(x, z))) $$
      \item[\textbf{iii)}] Leichtschläfer haben nichts, was nachts heult.
            $$ \forall x (LS(x) \rightarrow \neg \exists y (Hat(x, y) \land Heult(y))) $$
      \item[\textbf{iv)}] John hat eine Katze oder einen Hund.
            $$ \exists x (Hat(John, x) \land (Katze(x) \lor Hund(x))) $$
      \item[\textbf{Ziel:}] Falls John ein Leichtschläfer ist, hat er keine Mäuse.
            $$ LS(John) \rightarrow \neg \exists z (Maus(z) \land Hat(John, z)) $$
\end{enumerate}

\subsection*{2b) Umformung in Klauselnormalform (CNF)}

Wir negieren zunächst die zu beweisende Schlussfolgerung für den Widerspruchsbeweis.

\paragraph{Negierte Schlussfolgerung:}
Negation von $A \rightarrow B$ ist $A \land \neg B$.
$$ \neg (LS(John) \rightarrow \neg \exists z (Maus(z) \land Hat(John, z))) $$
$$ \equiv LS(John) \land \exists z (Maus(z) \land Hat(John, z)) $$
Skolemisierung: Wir ersetzen $\exists z$ durch die Skolem-Konstante $m_{aus}$.
\begin{itemize}
      \item $K_1: \{ LS(John) \}$
      \item $K_2: \{ Maus(m_{aus}) \}$
      \item $K_3: \{ Hat(John, m_{aus}) \}$
\end{itemize}

\paragraph{Umformung der Axiome:}

\textbf{Zu i) Hunde heulen:}
$$ \forall x (\neg Hund(x) \lor Heult(x)) $$
\begin{itemize}
      \item $K_4: \{ \neg Hund(x_1), Heult(x_1) \}$
\end{itemize}

\textbf{Zu ii) Katzenbesitzer haben keine Mäuse:}
Implikation auflösen ($A \rightarrow B \equiv \neg A \lor B$):
$$ \forall x (\neg (\exists y (Katze(y) \land Hat(x, y))) \lor \neg (\exists z (Maus(z) \land Hat(x, z)))) $$
Negation reinziehen (De Morgan \& Quantorenwechsel):
$$ \forall x ((\forall y (\neg Katze(y) \lor \neg Hat(x, y))) \lor (\forall z (\neg Maus(z) \lor \neg Hat(x, z)))) $$
Variablen umbenennen und Quantoren droppen:
\begin{itemize}
      \item $K_5: \{ \neg Katze(y_2), \neg Hat(x_2, y_2), \neg Maus(z_2), \neg Hat(x_2, z_2) \}$
\end{itemize}

\textbf{Zu iii) Leichtschläfer:}
$$ \forall x (\neg LS(x) \lor \neg \exists y (Hat(x, y) \land Heult(y))) $$
$$ \forall x (\neg LS(x) \lor \forall y (\neg Hat(x, y) \lor \neg Heult(y))) $$
\begin{itemize}
      \item $K_6: \{ \neg LS(x_3), \neg Hat(x_3, y_3), \neg Heult(y_3) \}$
\end{itemize}

\textbf{Zu iv) Johns Tier:}
Skolemisierung von $\exists x$: Wir ersetzen $x$ durch Skolem-Konstante $a$ (Animal).
$$ Hat(John, a) \land (Katze(a) \lor Hund(a)) $$
Dies ergibt zwei Klauseln:
\begin{itemize}
      \item $K_7: \{ Hat(John, a) \}$
      \item $K_8: \{ Katze(a), Hund(a) \}$
\end{itemize}

\subsection*{2c) Resolutionsbeweis}

Wir versuchen, die leere Klausel $\square$ abzuleiten.

\begin{enumerate}
      \item \textbf{Schritt 1:} Johns Tier $a$ heult nicht (aus Leichtschläfer-Eigenschaft). \\
            Resolviere $K_6$ und $K_1$:
            $$ \{\neg LS(x_3), \neg Hat(x_3, y_3), \neg Heult(y_3)\} + \{LS(John)\} $$
            Unifikation: $\sigma = \{ x_3 / John \}$ \\
            $\Rightarrow R_1: \{ \neg Hat(John, y_3), \neg Heult(y_3) \}$

      \item \textbf{Schritt 2:} Verknüpfung mit Johns Tier $a$. \\
            Resolviere $R_1$ und $K_7$:
            $$ \{\neg Hat(John, y_3), \neg Heult(y_3)\} + \{Hat(John, a)\} $$
            Unifikation: $\sigma = \{ y_3 / a \}$ \\
            $\Rightarrow R_2: \{ \neg Heult(a) \}$

      \item \textbf{Schritt 3:} Tier $a$ ist kein Hund (da Hunde heulen). \\
            Resolviere $R_2$ und $K_4$:
            $$ \{\neg Heult(a)\} + \{\neg Hund(x_1), Heult(x_1)\} $$
            Unifikation: $\sigma = \{ x_1 / a \}$ \\
            $\Rightarrow R_3: \{ \neg Hund(a) \}$

      \item \textbf{Schritt 4:} Tier $a$ ist eine Katze (Disjunktion auflösen). \\
            Resolviere $R_3$ und $K_8$:
            $$ \{\neg Hund(a)\} + \{Katze(a), Hund(a)\} $$
            Keine Unifikation nötig. \\
            $\Rightarrow R_4: \{ Katze(a) \}$

      \item \textbf{Schritt 5:} Widerspruch zur Maus (Katzenbesitzer-Regel). \\
            Resolviere $R_4$ und $K_5$:
            $$ \{Katze(a)\} + \{ \neg Katze(y_2), \neg Hat(x_2, y_2), \neg Maus(z_2), \neg Hat(x_2, z_2) \} $$
            Unifikation: $\sigma = \{ y_2 / a \}$ \\
            $\Rightarrow R_5: \{ \neg Hat(x_2, a), \neg Maus(z_2), \neg Hat(x_2, z_2) \}$

      \item \textbf{Schritt 6:} Spezialisierung auf John (da John $a$ besitzt). \\
            Resolviere $R_5$ und $K_7$:
            $$ \{ \neg Hat(x_2, a), \neg Maus(z_2), \neg Hat(x_2, z_2) \} + \{ Hat(John, a) \} $$
            Unifikation: $\sigma = \{ x_2 / John \}$ \\
            $\Rightarrow R_6: \{ \neg Maus(z_2), \neg Hat(John, z_2) \}$ \\
            (Bedeutung: John besitzt keine Maus).

      \item \textbf{Schritt 7:} Kollision mit der Annahme, John habe Maus $m_{aus}$. \\
            Resolviere $R_6$ und $K_2$:
            $$ \{ \neg Maus(z_2), \neg Hat(John, z_2) \} + \{ Maus(m_{aus}) \} $$
            Unifikation: $\sigma = \{ z_2 / m_{aus} \}$ \\
            $\Rightarrow R_7: \{ \neg Hat(John, m_{aus}) \}$

      \item \textbf{Schritt 8:} Abschluss. \\
            Resolviere $R_7$ und $K_3$:
            $$ \{ \neg Hat(John, m_{aus}) \} + \{ Hat(John, m_{aus}) \} $$
            $\Rightarrow \square$ (Leere Klausel)
\end{enumerate}

Da die leere Klausel abgeleitet werden konnte, ist der Widerspruchsbeweis erbracht. Die ursprüngliche Behauptung ist wahr.

\end{document}