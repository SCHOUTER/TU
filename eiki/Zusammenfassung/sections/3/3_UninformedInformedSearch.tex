\documentclass[../../eiki_summary.tex]{subfiles}

\externaldocument[ext:]{../../eiki_summary}
% Set Graphics Path, so pictures load correctly
\graphicspath{{../../pics/}}
% TODO: für jede Suchart ein Bsp./Grafik darstellen

\begin{document}

\section{Uninformed and Informed Search}

\subsection{Problem Formulation}

Problem-solving agents are result-driven. They always focus on satisfying their goals, i.e., solving the problem. While problems are often given in a human-understandable way, we need to reformulate the problem for our agent. These agents employ algorithms to find solutions.

Steps to formulate a solvable problem:
\begin{enumerate}
  \item \textbf{Formulate the goal}
  \item \textbf{Formulate the problem} given the goal
\end{enumerate}

\subsubsection{Key Terminology}

\begin{defbox}[The State Space / States]
  A state describes a possible situation in our environment. The \textbf{state space} is a set of all possible situations (states).
\end{defbox}

\begin{defbox}[Transition / Action]
  Transitions describe possible actions to take between one state and another. We only count direct transitions between two states (single actions).
\end{defbox}

\begin{defbox}[Costs]
  Often transitions aren't alike and differ. We express this by adding a "cost" to each action. Often the goal in search algorithms is to \textbf{minimize the cost} to reach the goal.
\end{defbox}

A \textbf{single state problem} is defined by 4 items:
\begin{enumerate}
  \item \textbf{State space and Initial state} \
        Description of all possible states and the initial environment as state.
  \item \textbf{Description of actions} \
        Typically a function that maps a state to a set of possible actions in this state.
  \item \textbf{Goal test} \
        Typically a function to test if the current state fulfills the goal definition.
  \item \textbf{Costs} \
        A cost function that maps actions to its cost. An easy way is to have additive costs (sum of costs for all actions taken).
\end{enumerate}

\subsubsection{The State-Space Graph}
The state space is the set of all states reachable from the initial state. It is implicitly defined by the initial state and the successor function, forming a \textbf{state-space graph}.

\begin{itemize}
  \item \textbf{Path:} A sequence of states connected by a sequence of actions.
  \item \textbf{Solution:} A path that leads from the initial state to a goal state.
  \item \textbf{Optimal Solution:} A solution with the minimum path cost.
\end{itemize}

\subsubsection{Core Search Definitions}
\begin{defbox}[Planning Problem]
  A planning problem is one in which we have an initial state and want to transform it into a desired goal, considering future actions and their outcomes.
\end{defbox}

\begin{defbox}[Search]
  The process of finding the (optimal) solution for such a problem in the form of a sequence of actions.
\end{defbox}

\subsection{Search Fundamentals}

\subsubsection{Tree Search vs. Graph Search}
The state-space graph can be explored by building a \textbf{search tree}.

\begin{itemize}
  \item \textbf{Tree Search:} Treats the state space as a tree. It does not keep track of visited states, so it might re-explore the same state via a different path. This can lead to exponential work for problems with loops or redundant paths.
  \item \textbf{Graph Search:} Remembers states that have been visited in an \textbf{explored set} (or "closed set"). It avoids expanding states that are already in the explored set, thus handling loops and redundant paths efficiently.
\end{itemize}

\subsubsection{States vs. Nodes}
\begin{defbox}[State]
  Representation of a physical configuration. Describes a specific situation in our environment (e.g., "in Arad").
\end{defbox}

\begin{defbox}[Node]
  A data structure to represent a part of a search tree. It includes a \textbf{state}, a \textbf{parent node}, the \textbf{action} taken, the \textbf{path cost} ($g(n)$), and the \textbf{depth} (e.g., "the path Arad $\rightarrow$ Sibiu").
\end{defbox}

\subsubsection{Key Search Tree Terminology}
\begin{defbox}[Fringe]
  The set of all nodes at the end of all visited paths is called the fringe. (Also known as \textbf{frontier} or "open set"). These are the nodes available for expansion.
\end{defbox}

\begin{defbox}[Depth]
  Number of levels in the search tree.
\end{defbox}

\subsubsection{Evaluating Search Strategies}
Search strategies are evaluated along the following dimensions:
\begin{itemize}
  \item \textbf{Completeness:} Does it always find a solution if one exists?
  \item \textbf{Time Complexity:} Number of node expansions.
  \item \textbf{Space Complexity:} Maximum number of nodes in memory.
  \item \textbf{Optimality:} Does it always find the optimal (least-cost) solution?
\end{itemize}
Complexity is measured in terms of:
\begin{itemize}
  \item $b$: maximum \textbf{branching factor} of the search tree.
  \item $d$: the \textbf{depth} of the optimal solution.
  \item $m$: the \textbf{maximum depth} of the state space (may be $\infty$).
\end{itemize}

\subsection{Uninformed Search Strategies}

\begin{defbox}[Uninformed Search]
  Do not have any information about the problem except the problem definition. (Also called \textbf{Blind Search}).
\end{defbox}

\begin{defbox}[Breadth-First Search (BFS)]
  A special case of Uniform-Cost Search where all step costs are equal. It starts at the tree root and explores the tree \textbf{level by level}. It uses a FIFO (First-In-First-Out) queue for the fringe.
\end{defbox}
\begin{itemize}
  \item \textbf{Completeness:} Yes.
  \item \textbf{Time:} $O(b^d)$ (The summary table uses $O(b^{d+1})$).
  \item \textbf{Space:} $O(b^d)$. Memory consumption is its biggest drawback.
  \item \textbf{Optimality:} Yes (if all costs are equal).
\end{itemize}

\begin{defbox}[Uniform-Cost Search (UCS)]
  Each node is associated with a cost, which accumulates over the path. UCS expands the node with the \textbf{lowest cumulative path cost} ($g(n)$). It is often implemented with a \textbf{priority queue}.
\end{defbox}
\begin{itemize}
  \item \textbf{Completeness:} Yes (if step costs are positive, i.e., $>\epsilon > 0$).
  \item \textbf{Time:} $O(b^(1+\lfloor C^/\epsilon \rfloor))$, where $C^*$ is the cost of the optimal solution.
  \item \textbf{Space:} $O(b^(1+\lfloor C^*/\epsilon \rfloor))$.
  \item \textbf{Optimality:} Yes.
\end{itemize}

\begin{defbox}[Depth-First Search (DFS)]
  Starts at the tree root and explores as far as possible along one branch before backtracking. It uses a LIFO (Last-In-First-Out) stack for the fringe.
\end{defbox}
\begin{itemize}
  \item \textbf{Completeness:} No. Fails in infinite-depth spaces or spaces with loops.
  \item \textbf{Time:} $O(b^m)$, where $m$ is the max depth. Can be terrible if $m \gg d$.
  \item \textbf{Space:} $O(b \times m)$. This linear space complexity is its key advantage.
  \item \textbf{Optimality:} No.
\end{itemize}

\begin{defbox}[Depth-limited Search (DLS)]
  A variation of DFS where the search is limited to a predetermined depth $l$. Nodes at depth $l$ are not expanded.
\end{defbox}
\begin{itemize}
  \item \textbf{Completeness:} No (if $l < d$).
  \item \textbf{Time:} $O(b^l)$.
  \item \textbf{Space:} $O(b \times l)$.
  \item \textbf{Optimality:} No.
\end{itemize}

\begin{defbox}[Iterative Deepening Search (IDS)]
  Combines the benefits of BFS and DFS. It runs DLS repeatedly with increasing depth limits: $l=0, 1, 2, \dots, d$.
\end{defbox}
\begin{itemize}
  \item \textbf{Completeness:} Yes.
  \item \textbf{Time:} $O(b^d)$ (Despite re-generating upper levels, the overhead is small).
  \item \textbf{Space:} $O(b \times d)$.
  \item \textbf{Optimality:} Yes (if costs are uniform).
\end{itemize}

\begin{defbox}[Bidirectional Search]
  Performs two searches simultaneously: one forward from the initial state, one backward from the goal state. Stops when the two searches meet.
\end{defbox}
\begin{itemize}
  \item \textbf{Completeness:} Yes.
  \item \textbf{Time:} $O(b^{d/2})$.
  \item \textbf{Space:} $O(b^{d/2})$.
  \item \textbf{Notes:} Only possible if actions can be reversed.
\end{itemize}

\subsubsection{Summary of Uninformed Strategies}
\begin{table}[h!]
  \centering
  \renewcommand{\arraystretch}{1.3}
  \setlength{\tabcolsep}{4pt}
  \begin{tabular}{lccccc}
    \hline
    \textbf{Criterion} & \textbf{Breadth-First} & \textbf{Uniform-Cost}               & \textbf{Depth-First} & \textbf{Depth-Limited} & \textbf{Iterative Deepening} \\
    \hline
    Complete?          & Yes$^*$                & Yes$^*$                             & No                   & Yes, if $l \ge d$      & Yes                          \\
    Time               & $O(b^{d+1})$           & $O(b^{\lceil C^*/\epsilon \rceil})$ & $O(b^m)$             & $O(b^l)$               & $O(b^d)$                     \\
    Space              & $O(b^{d+1})$           & $O(b^{\lceil C^*/\epsilon \rceil})$ & $O(bm)$              & $O(bl)$                & $O(bd)$                      \\
    Optimal?           & Yes$^*$                & Yes                                 & No                   & No                     & Yes$^*$                      \\
    \hline
  \end{tabular}
  \caption{Comparison of uninformed search strategies. ($^*$Assumes uniform step costs or $l \ge d$ where applicable).}
\end{table}

\subsection{Informed Search Strategies}

\begin{defbox}[Informed Search]
  Have additional knowledge about the problem (beyond the definition) and an idea of where to "look" for solutions.
\end{defbox}
This "hint" is provided by a heuristic function.

\subsubsection{Heuristics}
\begin{defbox}[Heuristic $h(n)$]
  Informally denotes a "rule of thumb". In tree-search, a heuristic is a function $h(n)$ that \textbf{estimates the remaining cost} to reach the goal from node $n$.
\end{defbox}

\subsubsection{Greedy Best-first Search}
\begin{defbox}[Greedy Best-first Search]
  Uses an evaluation function $f(n) = h(n)$ to estimate the cost from node $n$ to the goal. It expands the node that appears to be closest to the goal, according to the heuristic. It does not care about the actual cost/distance.
\end{defbox}
\begin{itemize}
  \item \textbf{Completeness:} No. Can get stuck in loops. (Complete in finite spaces with loop detection).
  \item \textbf{Time:} Worst case $O(b^m)$.
  \item \textbf{Space:} Worst case $O(b^m)$ (keeps all nodes in memory).
  \item \textbf{Optimality:} No. The solution depends entirely on the heuristic.
\end{itemize}

\subsubsection{A* Search}
\begin{defbox}[A* Search]
  An informed tree search algorithm, building on best-first search. It is the "best-known" form. It avoids expanding paths that are already expensive.
\end{defbox}
A* evaluates nodes using the function: $f(n) = g(n) + h(n)$.
\begin{itemize}
  \item $g(n) = $ \textbf{true cost} so far to reach node $n$.
  \item $h(n) = $ \textbf{estimated cost} to get from $n$ to the goal.
  \item $f(n) = $ \textbf{estimated cost} of the cheapest solution path that goes through node $n$.
\end{itemize}
\begin{itemize}
  \item \textbf{Completeness:} Yes (unless there are infinitely many nodes with $f(n) \le f(G)$).
  \item \textbf{Time:} Can be exponential unless the error of the heuristic $h(n)$ is bounded.
  \item \textbf{Space:} Has to keep all nodes in memory. This is the primary drawback of A*.
  \item \textbf{Optimality:} Yes, \textbf{if the heuristic $h(n)$ is admissible}.
\end{itemize}

\subsubsection{Heuristic Properties}

\begin{defbox}[Admissible Heuristics]
  A heuristic is \textbf{admissible} if it \textbf{never overestimates} the true cost to reach a goal.
  Formally:
  \[
    h(n) \le h^*(n) \quad \text{for all nodes } n,
  \]
  where $h(n)$ is the heuristic estimate and $h^*(n)$ is the actual optimal cost from $n$ to the goal.
  (For example, the straight-line distance heuristic $h_{\text{SLD}}$ is admissible in route-finding problems.)
\end{defbox}

\begin{defbox}[Consistent Heuristics]
  A heuristic is \textbf{consistent} if for every node $n$ and every successor $n'$ generated by action $a$, the "triangle inequality" holds:
  $h(n) \le c(n, a, n') + h(n')$.
  This means the heuristic difference between adjacent nodes never overestimates the actual step cost.
\end{defbox}
\begin{itemize}
  \item \textbf{Lemma 1:} Every \textbf{consistent} heuristic is also \textbf{admissible}.
  \item \textbf{Lemma 2:} If $h(n)$ is consistent, then the values of $f(n)$ along any path are \textbf{non-decreasing}.
\end{itemize}

\begin{defbox}[Relaxed Problems]
  A problem with fewer restrictions on the actions is called a relaxed problem. The cost of an optimal solution to a relaxed problem is an \textbf{admissible heuristic} for the original problem.
\end{defbox}
Example (8-puzzle):
\begin{itemize}
  \item $h_1(n) = $ Number of misplaced tiles. (Relaxed rule: tile can move anywhere).
  \item $h_2(n) = $ Total Manhattan distance. (Relaxed rule: tile can move to any adjacent square).
  \item Both $h_1$ and $h_2$ are admissible.
\end{itemize}

\begin{defbox}[Dominance]
  If $h_1$ and $h_2$ are both admissible and $h_2(n) \ge h_1(n)$ for all $n$, then $h_2$ \textbf{dominates} $h_1$.
\end{defbox}
A* will expand fewer nodes with a dominant heuristic. (e.g., for the 8-puzzle, $h_2$ (Manhattan) dominates $h_1$ (misplaced tiles)).

\paragraph{Combining Heuristics}
If we have several admissible heuristics $h_1(n), \dots, h_m(n)$, we can combine them.
$h(n) = \max{h_1(n), h_2(n), \dots, h_m(n)}$ is also admissible and dominates all of its components.

\subsubsection{Optimality of A}
A (using Tree Search) is optimal if its heuristic $h(n)$ is admissible.

\paragraph{Proof (Informal):}
\begin{enumerate}
  \item Let $G$ be an optimal goal state, with path cost $C^*$.
  \item Assume for contradiction that A is about to return a suboptimal goal $G_2$, with path cost $g(G_2) > C^*$.
  \item At this moment, $G_2$ is in the fringe. Because $A^*$ chose $G_2$, its $f$-value must be the lowest, so $f(G_2) \le f(n)$ for all other fringe nodes $n$.
  \item Let $n$ be any unexpanded node on a true optimal path to $G$. This node $n$ must be in the fringe.
  \item \textbf{Analyze $f(G_2)$:}
        For a goal state, $h(G_2) = 0$.
        So, $f(G_2) = g(G_2) + h(G_2) = g(G_2)$.
        Since $G_2$ is suboptimal, $f(G_2) = g(G_2) > C^*$.
  \item \textbf{Analyze $f(n)$:}
        $f(n) = g(n) + h(n)$.
        Because $h$ is admissible, $h(n) \le h^(n)$ (where $h^(n)$ is the true cost from $n$ to $G$).
        The true cost of the optimal path is $C^ = g(n) + h^(n)$.
        Therefore, $f(n) = g(n) + h(n) \le g(n) + h^(n) = C^*$.
  \item \textbf{Contradiction:}
        We have shown $f(n) \le C^*$ and $f(G_2) > C^*$.
        This means $f(n) < f(G_2)$.
        A would be forced to expand $n$ (on the optimal path) \textit{before} it could ever expand $G_2$.
        Thus, A* can never select a suboptimal goal. It is optimal.
\end{enumerate}

\subsubsection{Memory-Bounded Heuristic Search}
The main problem with A* is its space complexity. Alternatives include:
\begin{enumerate}
  \item \textbf{Iterative-deepening A* (IDA):} Like IDS, but the "depth" cutoff is the $f$-cost ($g+h$).
  \item \textbf{Recursive best-first search (RBFS):} Mimics best-first search using linear space by recursively re-expanding nodes and updating $f$-values from ancestors.
  \item \textbf{(Simple) Memory-bounded A ((S)MA*):} When memory is full, drops the worst (highest $f$-value) leaf node.
\end{enumerate}

\subsection{Tree Search vs. Graph Search (Revisited)}

Failure to detect repeated states can turn a linear problem into an exponential one!

\begin{defbox}[Graph Search]
  Uses an \textbf{explored set} (or "closed set") to store all states that have been expanded. When expanding a node, its successors are only added to the fringe \textbf{if they are not in the fringe or explored set}.
\end{defbox}

\paragraph{Optimality of A* Graph Search}
\begin{itemize}
  \item If $h(n)$ is only \textbf{admissible}, Graph Search A* is not guaranteed to be optimal. It might find a suboptimal path to a node first, add it to the explored set, and never find the optimal path.
  \item If $h(n)$ is \textbf{consistent}, Graph Search A* \textbf{is optimal}.
  \item \textbf{Why?} A consistent heuristic guarantees that $f$-values are non-decreasing along any path. This means the \textit{first} time A* expands a node $n$, it is \textit{guaranteed} to have found the shortest possible path to it. Therefore, we never need to re-expand any node in the explored set.
\end{itemize}

\end{document}