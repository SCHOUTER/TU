\documentclass[11pt,a4paper]{article}
\title{\"Ubung 4}
\author{Niclas Kusenbach, 360227}

\usepackage[T1]{fontenc}
\usepackage[utf8]{inputenc}
\usepackage[ngerman,british]{babel}
\usepackage{amsmath,amsfonts,amssymb}
\usepackage{booktabs}
\usepackage{graphicx}
\usepackage{array}
\usepackage{siunitx}
\usepackage{physics}
\usepackage{xcolor}
\usepackage{enumitem}
\usepackage[final]{microtype} %Fix most of overfull hboxes
\usepackage{xurl} % better line breaks for \url and plain URLs
\usepackage{hyperref} % better line breaks for \url and plain URLs
\usepackage{tabularx}
\setlist{nosep}

\begin{document}

\maketitle

\section*{Aufgabe 1}

\subsection*{Aufgabe 1a: Formalisierung als CSP}

\textbf{Variablen:}
Die Menge der Variablen ist $V = \{A, B, C, U\}$.
\begin{itemize}
  \item $A, B, C$: Repräsentieren die Ziffern der Gleichung.
  \item $U$: Repräsentiert den Übertrag (Carry) von der Einer- in die Zehnerstelle.
\end{itemize}
\textbf{Wertebereiche (Domains):}
\begin{itemize}
  \item $D_A = \{0, 1, 2, 3, 4, 5, 6, 7, 8, 9\}$
  \item $D_B = \{1, 2, 3, 4, 5, 6, 7, 8, 9\}$ \quad (Da $B$ führende Ziffer im Ergebnis ist, gilt $B \neq 0$)
  \item $D_C = \{0, 1, 2, 3, 4, 5, 6, 7, 8, 9\}$
  \item $D_U = \{0, 1\}$ \quad (Maximaler Übertrag bei $9+9$ ist $1$)
\end{itemize}
\textbf{Constraints:}
\begin{enumerate}
  \item \textbf{Alldiff-Constraint:} Jeder Buchstabe steht für eine unterschiedliche Ziffer.
        \[ A \neq B \land A \neq C \land B \neq C \]
  \item \textbf{Spalten-Constraints (Addition):}
        \begin{itemize}
          \item \textit{Einerstelle (Rechts):} $A + B = 10 \cdot U + C$
          \item \textit{Zehnerstelle (Links):} $U = B$
        \end{itemize}
\end{enumerate}

\vspace{0.5cm}

\subsection*{Aufgabe 1b: Backtracking-Search}

\textbf{Annahmen:}
\begin{itemize}
  \item Variablenreihenfolge: $C, B, A, U$.
  \item Werteauswahl: Aufsteigend ($0 \to 9$).
\end{itemize}

\textbf{Ablauf:}

\begin{enumerate}
  \item \textbf{Variable C:} Wähle $C = 0$.
        \begin{itemize}
          \item Konsistenzprüfung: OK.
        \end{itemize}

  \item \textbf{Variable B:} Wähle $B = 1$ (0 nicht in $D_B$).
        \begin{itemize}
          \item Konsistenzprüfung ($alldiff$): $1 \neq 0$ $\rightarrow$ OK.
        \end{itemize}

  \item \textbf{Variable A:} Iteration durch $D_A = \{0, \dots, 9\}$.
        \begin{itemize}
          \item Versuch $A = 0$: Konflikt mit $C$ ($alldiff$). $\rightarrow$ Backtrack.
          \item Versuch $A = 1$: Konflikt mit $B$ ($alldiff$). $\rightarrow$ Backtrack.
          \item Versuch $A = 2$: $alldiff$ OK. $\rightarrow$ Weiter zu $U$.
                \begin{enumerate}
                  \item \textbf{Variable U:}
                  \item Versuch $U = 0$: Constraint $U=B$ ($0=1$) verletzt.
                  \item Versuch $U = 1$: Constraint $U=B$ ($1=1$) OK. \\
                        Aber Addition: $A+B = 10U+C \Rightarrow 2+1 \neq 10(1)+0 \Rightarrow 3 \neq 10$. Verletzt.
                  \item Keine Werte mehr für $U$. $\rightarrow$ Backtrack zu $A$.
                \end{enumerate}
          \item Versuche $A = 3$ bis $A = 8$: Scheitern analog im Schritt $U$ an der Summenbedingung.
          \item Versuch $A = 9$: $alldiff$ OK. $\rightarrow$ Weiter zu $U$.
                \begin{enumerate}
                  \item \textbf{Variable U:}
                  \item Versuch $U = 0$: Verletzt $U=B$.
                  \item Versuch $U = 1$: Erfüllt $U=B$ ($1=1$). \\
                        Prüfung Addition: $9+1 = 10(1) + 0 \Rightarrow 10 = 10$. OK.
                \end{enumerate}
        \end{itemize}
\end{enumerate}

\textbf{Ergebnis:} Gefundene Belegung $\sigma = \{C=0, B=1, A=9, U=1\}$.

\vspace{0.5cm}

\subsection*{Aufgabe 1c: Forward-Checking und Constraint-Propagation}

\textbf{Einsatz von Forward-Checking (FC):}
Forward-Checking eliminiert inkonsistente Werte aus den Domains der \textit{noch nicht zugewiesenen} Variablen.

\begin{enumerate}
  \item Zuweisung $C = 0$: Entfernt 0 aus $D_A$ und $D_B$.
  \item Zuweisung $B = 1$:
        \begin{itemize}
          \item Entfernt 1 aus $D_A$.
          \item Durch Constraint $U=B$ reduziert sich $D_U$ auf $\{1\}$.
        \end{itemize}
  \item Zuweisung $A$:
        \begin{itemize}
          \item Bei Wahl von $A=2$: FC prüft Konsistenz mit verbleibender Variable $U$.
          \item Gleichung $A+B = 10U+C$ ergibt $2+1 = 10U \Rightarrow U=0.3$.
          \item Der Wert $0.3$ (bzw. 0) ist nicht in $D_U = \{1\}$. Die Domain $D_U$ läuft leer (Domain Wipeout).
          \item \textbf{Effekt:} Der Algorithmus bricht für $A=2$ (und analog $A=3 \dots 8$) sofort ab, ohne die Variable $U$ überhaupt zu instanziieren.
        \end{itemize}
\end{enumerate}
\textbf{Potenzial von Constraint-Propagation höherer Ordnung (Arc Consistency):}
Forward-Checking prüft nur die Konsistenz zwischen der \textit{aktuellen} Variable und den \textit{zukünftigen}. Es prüft nicht die Beziehungen zwischen zwei zukünftigen Variablen (hier $A$ und $U$).

\begin{itemize}
  \item Nach der Zuweisung $C=0$ und $B=1$ sind $A$ und $U$ noch offen.
  \item $D_A = \{2, \dots, 9\}$ und $D_U = \{1\}$.
  \item Eine Prüfung der Kantenkonsistenz (Arc Consistency, z.B. AC-3) hätte erkannt:
        \[ A + 1 = 10 \cdot U + 0 \]
        Da $U$ zwingend 1 ist (wegen $B=1$), muss $A + 1 = 10$ sein, also $A=9$.
  \item \textbf{Reduktion:} Alle Werte außer 9 wären aus $D_A$ entfernt worden, \textbf{bevor} die Suche für $A$ überhaupt begonnen hätte. Dies hätte die Fehlversuche für $A \in \{2..8\}$ komplett vermieden (Backtrack-free search in diesem Zweig).
\end{itemize}

\end{document}