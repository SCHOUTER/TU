\documentclass[11pt,a4paper]{article}
\title{\"Ubung 4}
\author{Niclas Kusenbach, 360227}

\usepackage[T1]{fontenc}
\usepackage[utf8]{inputenc}
\usepackage[ngerman,british]{babel}
\usepackage{amsmath,amsfonts,amssymb}
\usepackage{booktabs}
\usepackage{graphicx}
\usepackage{array}
\usepackage{siunitx}
\usepackage{physics}
\usepackage{xcolor}
\usepackage{enumitem}
\usepackage[final]{microtype} %Fix most of overfull hboxes
\usepackage{xurl} % better line breaks for \url and plain URLs
\usepackage{hyperref} % better line breaks for \url and plain URLs
\usepackage{tabularx}
\setlist{nosep}

\begin{document}

\maketitle

\section*{1) Lokale Suchalgorithmen}

\subsection*{a)}
Da man immer nur den aktuellsten besten Nachfolger verfolgt, gibt es keine parallelen Strahlen. \textbf{Hill Climb}
\subsection*{b)}
Eine Temperatur nahe 0 bedeutet, dass schlechtere Zustände, praktisch nie akzeptiert werden. \textbf{Hill Climb}
\subsection*{c)}
Wenn die Temperatur unendlich ist werden schlechtere Zustände immer akzeptiert. Damit verhält sich das Verfahren wie ein Zufallsprozess: \textbf{Random Walk.}

\section*{2) Spielbaumsuche}

\subsection*{a)}
Die Lösung ist $3 \leq x < 7$. \newline
Damit x gegen den rechten Teilbaum gewinnt muss es größer oder gleich 3 sein. Damit es gegen die 7 gewinnt, kleiner als diese.

\subsection*{b)}

\begin{enumerate}
  \item Der linke Teilbaum legt den Alpha-Wert der Wurzel fest auf das Ergebnis von Knoten (b), also $\alpha = \min(7, x)$.
  \item Im rechten Teilbaum liefert der erste geprüfte Ast (Knoten f) den Wert 3, wodurch der MIN-Knoten (c) sein Beta auf $\beta = 3$ setzt.
  \item Der Schnitt (Pruning) erfolgt, wenn $\alpha \ge \beta$ gilt.
  \item Setzt man die Werte ein ($\min(7, x) \ge 3$), muss folglich $x \ge 3$ sein, damit der Rest (Knoten g) ignoriert wird.
\end{enumerate}

\end{document}