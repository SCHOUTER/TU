\documentclass[
../../css_summary.tex,
]
{subfiles}

\externaldocument[ext:]{../../css_summary}
% Set Graphics Path, so pictures load correctly
\graphicspath{{../../pics/}}

\begin{document}

\section{Public Key Infrastrukturen (PKI)}

Verschlüsselung allein reicht für sichere Kommunikation nicht aus. Ein Angreifer könnte sich in den Kanal einklinken (Man-in-the-Middle). Daher ist die \defc{Authentifikation} des Kommunikationspartners essenziell.

\begin{itemize}
  \item \textbf{Das Problem:} Woher weiß ich, dass der öffentliche Schlüssel (Public Key), den ich erhalte, wirklich zu der Person/Webseite gehört, mit der ich kommunizieren will?
  \item \textbf{Die Lösung:} Eine Infrastruktur, die Schlüssel an Identitäten bindet.
\end{itemize}

\subsection{Vertrauensmodelle (Trust Models)}

Es gibt drei fundamentale Ansätze, um Vertrauen in öffentliche Schlüssel zu etablieren.

\subsubsection{1. Direct Trust}
Dies ist das einfachste Modell, bei dem Schlüssel direkt zwischen den Parteien ausgetauscht werden.

\begin{defbox}[Direct Trust]
  Direkter, manueller Austausch von öffentlichen Schlüsseln oder Fingerprints. Vertrauen entsteht durch persönliche Überprüfung.
\end{defbox}

\begin{itemize}
  \item \textbf{Beispiel SSH:} Beim ersten Verbinden ("Trust on First Use" - TOFU) zeigt der Client den Fingerprint des Servers.
  \item \textit{Warnmeldung:} "The authenticity of host... can't be established." Der Nutzer muss den Fingerprint manuell verifizieren (z.B. über einen sicheren zweiten Kanal).
  \item \textbf{Nachteil:} Skaliert nicht. Man kann nicht mit jedem Webseiten-Betreiber der Welt persönlich Schlüssel tauschen.
\end{itemize}

\subsubsection{2. Hierarchical Trust (WebPKI)}
Dieses Modell nutzt eine vertrauenswürdige dritte Partei, die als Mittelsmann fungiert. Dies ist der Standard im Web (HTTPS).

\begin{defbox}[Certificate Authority (CA)]
  Eine \defc{Zertifizierungsstelle} (CA) prüft die Identität eines Antragstellers und signiert dessen öffentlichen Schlüssel kryptographisch. Diese Signatur bildet ein Zertifikat.
\end{defbox}

\begin{itemize}
  \item \textbf{Funktionsweise:} Alice vertraut der CA. Die CA bürgt für Bob. Folglich vertraut Alice Bob.
  \item \textbf{Verteilung:} Die Root-Zertifikate der CAs sind im Betriebssystem oder Browser vorinstalliert (Trust Store).
\end{itemize}

\begin{center}
  \includegraphics[width=0.7\textwidth]{css_1_09_01PublicKeyInfrastructure_page_8_1.png}
\end{center}

\subsubsection{3. Web of Trust}
Ein dezentraler Ansatz, der oft bei PGP (E-Mail-Verschlüsselung) genutzt wird.

\begin{itemize}
  \item Es gibt keine zentralen CAs.
  \item Teilnehmer signieren gegenseitig ihre Schlüssel (\defc{Keysigning}).
  \item \textbf{Transitives Vertrauen:} Alice vertraut Bob. Bob hat Carols Schlüssel signiert. Wenn Alice Bobs Urteilsvermögen vertraut, vertraut sie auch Carol.
  \item \textbf{Keyserver:} Dienen als Telefonbuch zum Hochladen von Schlüsseln und Signaturen.
\end{itemize}

\subsection{WebPKI im Detail}

Das WebPKI-System bildet das Rückgrat des sicheren Browsings.

\subsubsection{Chain of Trust (Zertifikatskette)}
Browser vertrauen einer Menge an \defc{Root CAs} (Wurzelzertifikate). Eine Webseite sendet jedoch meist nicht das Root-Zertifikat, sondern eine Kette:
\begin{enumerate}
  \item \textbf{Root Certificate:} Selbstsigniert, im Browser hinterlegt (Trust Anchor).
  \item \textbf{Intermediate Certificate:} Von der Root CA (oder einer anderen Intermediate) signiert.
  \item \textbf{Leaf Certificate (End-Entity):} Das eigentliche Zertifikat der Webseite, signiert von der Intermediate CA.
\end{enumerate}
Der Browser validiert die Signaturen vom Leaf bis hoch zur Root.

\subsubsection{Validierungsmethoden für Zertifikate}
Bevor eine CA ein Zertifikat ausstellt, muss sie prüfen, ob der Antragsteller berechtigt ist.

\begin{enumerate}
  \item \textbf{Domain Validation (DV):}
        \begin{itemize}
          \item Prüft nur die technische Kontrolle über die Domain.
          \item \textit{Methoden:}
                \begin{itemize}
                  \item \textbf{HTTP Challenge:} CA gibt einen Token, Server muss ihn unter \texttt{http://domain/.well-known/acme-challenge/} bereitstellen.
                  \item \textbf{DNS Challenge:} Token muss als TXT-Record im DNS hinterlegt werden.
                  \item \textbf{Email Challenge:} Bestätigungslink an \texttt{admin@domain.com}.
                \end{itemize}
          \item \textit{Vorteil:} Schnell, automatisierbar (z.B. Let's Encrypt), kostenlos.
          \item \textit{Nachteil:} Keine Prüfung der Identität der Firma dahinter.
        \end{itemize}

  \item \textbf{Organization Validation (OV):} Prüft zusätzlich, ob die Organisation existiert und rechtmäßiger Besitzer der Domain ist.

  \item \textbf{Extended Validation (EV):} Sehr strenge Prüfung offizieller Dokumente/Register. Früher durch "grüne Adressleiste" im Browser angezeigt (heute meist entfernt, da Nutzer den Unterschied nicht verstehen).
\end{enumerate}

\subsection{Angriffe auf die WebPKI}

Das System ist nur so sicher wie das schwächste Glied (die CA) und der Validierungsprozess.

\subsubsection{Angriffe auf Domain Validation}
Da DV oft automatisiert abläuft, versuchen Angreifer, die Validierung (Challenge) zu manipulieren, um Zertifikate für fremde Domains zu erhalten.

\begin{itemize}
  \item \textbf{Netzwerk-Angriffe:} Wenn der Angreifer den Traffic zwischen der CA und dem Server des Opfers abfangen kann (Man-in-the-Middle).
  \item \textbf{BGP Hijacking:} Der Angreifer lenkt den Internetverkehr für die IP-Adresse des Opfers auf seinen eigenen Server um. Die CA verbindet sich zur Überprüfung (HTTP Challenge) mit dem Angreifer statt dem Opfer.
  \item \textbf{DNS Cache Poisoning:} Der Angreifer manipuliert die DNS-Antworten, die die CA erhält, sodass die Domain auf die IP des Angreifers zeigt.
\end{itemize}

\begin{center}
  \includegraphics[width=0.7\textwidth]{css_1_09_01PublicKeyInfrastructure_page_17_1.png}
  \includegraphics[width=0.7\textwidth]{css_1_09_01PublicKeyInfrastructure_page_18_1.png}
\end{center}

\paragraph{Gegenmaßnahme: Verteilte Validierung}
Die CA sollte die Validierung (z.B. den HTTP-Request) nicht nur von einem Standort ausführen, sondern von \defc{verteilten Validatoren} (mehrere Perspektiven weltweit). Ein lokaler BGP-Hijack oder DNS-Poisoning würde so auffallen, da nicht alle Validatoren auf den Angreifer umgeleitet werden.

\subsection{Certificate Transparency (CT)}

Ein großes Problem der klassischen PKI war, dass eine korrumpierte CA unbemerkt falsche Zertifikate (z.B. für google.com) ausstellen konnte.

\begin{defbox}[Certificate Transparency]
  Ein System öffentlich einsehbarer, manipulationssicherer Logs, in denen alle ausgestellten Zertifikate verzeichnet werden müssen.
\end{defbox}

\subsubsection{Funktionsweise}
\begin{enumerate}
  \item CA reicht ein "Pre-Certificate" bei einem CT-Log ein.
  \item Das Log antwortet mit einem \defc{SCT} (Signed Certificate Timestamp). Das ist eine kryptographische Quittung: "Ich habe dieses Zertifikat gesehen und werde es loggen."
  \item Das finale Zertifikat enthält diesen SCT.
  \item Browser (wie Chrome) lehnen Zertifikate ohne SCT oft ab.
\end{enumerate}

\subsubsection{Komponenten}
\begin{itemize}
  \item \textbf{Logs:} Nutzen \defc{Merkle Trees}, um Append-Only-Eigenschaften zu garantieren (Inhalte können nicht nachträglich gelöscht/geändert werden).
  \item \textbf{Monitors:} Dienste, die die Logs überwachen und Domain-Inhaber warnen, wenn unerwartet ein Zertifikat für ihre Domain auftaucht.
\end{itemize}

\begin{center}
  \includegraphics[width=0.7\textwidth]{css_1_09_01PublicKeyInfrastructure_page_22_1.png}
\end{center}

\subsection{Zertifikatsstruktur (X.509)}

Zertifikate folgen dem X.509 Standard und werden mittels \defc{ASN.1} (Abstract Syntax Notation One) beschrieben.

\subsubsection{Wichtige Felder}
\begin{itemize}
  \item \textbf{Subject:} Für wen ist das Zertifikat? (Common Name / Domain).
  \item \textbf{Issuer:} Wer hat es ausgestellt? (Die CA).
  \item \textbf{Validity:} \texttt{Not Before} und \texttt{Not After} (Gültigkeitsdauer).
  \item \textbf{Subject Public Key Info:} Der eigentliche Schlüssel und der Algorithmus (z.B. RSA, ECC).
  \item \textbf{Signature Algorithm:} Algorithmus, mit dem die CA unterschrieben hat (z.B. SHA256withRSA).
\end{itemize}

\subsubsection{Extensions}
\begin{itemize}
  \item \textbf{Key Usage / Basic Constraints:} Legt fest, was mit dem Zertifikat getan werden darf.
        \begin{itemize}
          \item Wichtig: \texttt{CA: FALSE} verhindert, dass ein normales Webseiten-Zertifikat genutzt wird, um weitere Zertifikate zu signieren (verhindert, dass jeder User zur CA wird).
        \end{itemize}
  \item \textbf{Subject Alternative Name (SAN):} Hier werden alle gültigen Domains (auch Subdomains) aufgelistet.
\end{itemize}

\subsubsection{Widerruf (Revocation) und Gültigkeit}
Wenn ein Private Key gestohlen wird, muss das Zertifikat ungültig gemacht werden.
\begin{itemize}
  \item \textbf{CRL (Certificate Revocation List):} Liste gesperrter Zertifikate. Wird oft nicht frisch heruntergeladen (zu groß).
  \item \textbf{OCSP (Online Certificate Status Protocol):} Live-Abfrage bei der CA. Problem: "Fail Open" (wenn CA nicht erreichbar, akzeptiert der Browser oft trotzdem).
  \item \textbf{Trend:} Verkürzung der Laufzeiten (Vorschlag von Google: 90 Tage). Ersetzt komplexes Revocation-Management durch häufige Erneuerung (Automation nötig).
\end{itemize}

\subsection{Alternative: DNSSEC / DANE}

Statt einer WebPKI mit hunderten CAs könnte man die Hierarchie des DNS nutzen.

\begin{itemize}
  \item \textbf{Idee:} Die "Chain of Trust" folgt der DNS-Delegation (. $\rightarrow$ .de $\rightarrow$ tu-darmstadt.de).
  \item \textbf{TLSA Records:} Der Fingerprint des Zertifikats wird direkt im DNS hinterlegt und per DNSSEC signiert.
  \item \textbf{Vorteil:} Kein "CA-Markt", logische Struktur.
  \item \textbf{Nachteil:} Ein einziger \defc{Root Key} (verwaltet von der ICANN) ist der "Single Point of Failure".
  \item \textbf{Root Signing Ceremony:} Hochsichere Zeremonie zur Verwaltung dieses Schlüssels (Safe deposit boxes, Zeugen, physische Sicherheit).
\end{itemize}

\subsection{Angriffe auf Nutzer (Typosquatting)}

Selbst bei perfekter Technik (valides Zertifikat) kann der Nutzer getäuscht werden, wenn er auf der falschen Seite landet.

\begin{itemize}
  \item \textbf{Homograph Attack:} Nutzung ähnlich aussehender Zeichen (z.B. kyrillisches 'a' statt lateinisches 'a').
  \item \textbf{Combo-Squatting:} Zusätze im Namen (\texttt{paypal-support.com}).
  \item \textbf{Homophon:} Ähnlich klingende Namen.
\end{itemize}
\textit{Hinweis:} Da WebPKI (DV) nur den Besitz der Domain prüft, erhält auch der Angreifer ein völlig valides, "grünes" Schloss-Symbol für seine Phishing-Seite \texttt{goggle.com}.
\end{document}