\documentclass[
../../css_summary.tex,
]
{subfiles}

\externaldocument[ext:]{../css_summary.tex}
% Set Graphics Path, so pictures load correctly
\graphicspath{{../}}

\begin{document}
\section{Kryptografie}
Die Kryptografie wird in drei Hauptkategorien unterteilt:
\begin{enumerate}
  \item Symmetrische Kryptografie
  \item Hashfunktionen
  \item Asymmetrische Kryptografie
\end{enumerate}

\subsection{Symmetrische Kryptografie}
Symmetrische Kryptografie ist eine Menge von kryptografischen Protokollen, bei der derselbe geheime Schlüssel für die Ver- und Entschlüsselung von Daten verwendet wird.

\begin{defbox}[Symmetrische Kryptosysteme]
Ein symmetrisches Kryptosystem ist ein 5-Tupel $(M, K, C, e, d)$ bestehend aus:
\begin{itemize}
  \item einer Menge $m \in M$ von Klartexten,
  \item einer Menge $k \in K$ von Schlüsseln,
  \item einer Menge $c \in C$ von Chiffretexten,
  \item einer Verschlüsselungsfunktion $e: M \times K \to C$,
  \item einer Entschlüsselungsfunktion $d: C \times K \to M$,
\end{itemize}
so dass für alle Klartexte $m \in M$ und alle Schlüssel $k \in K$ gilt, dass $d(e(m,k),k) = m$.
\end{defbox}

\subsubsection{Blockchiffren}
\begin{defbox}[Definition]
  Blockchiffren sind Kryptosysteme, die nur Blöcke fester Länge verschlüsseln können.
\end{defbox}
\begin{center}
  \includegraphics[width=200px]{../../pics/css_1_03_symm_krypto_RMU_page_5_1.png}
\end{center}
\begin{itemize}
  \item Ein Blockchiffre arbeitet auf einem Klartextblock der Länge $b$, um einen Chiffretextblock der Länge $b$ zu erzeugen.
  \item Der gleiche Schlüssel kann mehrmals auf unterschiedliche Blocks verwendet werden.
  \item Beispiele von Blockchiffren: AES, DES, 3DES, Serpent, Twofish, Blowfish, etc.
\end{itemize}
\textbf{Electronic Code Book (ECB) Modus}
\begin{center}
  \includegraphics[width=200px]{../../pics/css_1_03_symm_krypto_RMU_page_6_1.png}
  \includegraphics[width=200px]{../../pics/css_1_03_symm_krypto_RMU_page_7_1.png}
\end{center}
Wenn die Blöcke nicht die Länge $n$ haben, dann können trotzdem beliebige Nachrichten verschlüsselt werden, da eine \textbf{Auffüllungsfunktion} (Padding function) benutzt wird. Bspw.: $m* \to (m^n)*$
Bei Nachrichten, die die passenden Länge $n$ haben, sollte meistens $pad(x) = x$.
Eine Gute Auffüllfunktion sollte umkehrbar sein, d.h. es muss eine $unpad()$ funktion geben mit $unpad(pad(x)) = x \forall  x \in M*$
\defc{Vorteile:}
\begin{itemize}
  \item unkomplizierte Bedienung. Jeder Block wird unabhängig bearbeitet.
  \item Parallelisierbarkeit von Ver- und Entschlüsselungsverfahren
  \item Beschädigte Datenblöcke beeinflussen keine anderen Blöcke (Fehlertoleranz)
\end{itemize}
\defc{Nachteile:}
\begin{itemize}
  \item \textit{Deterministisch:} Muster im Klartext sind sichtbar. Identische Klartextblöcke ergeebn immer identische Chiffretextblöcke.
  \item \textit{Keine Diffusion:} Kleine Änderungen im Klartext führen zu lokalisierten Änderungen im Geheimtext.
\end{itemize}

\textbf{Cipher Block Chaining (CBC) Modus}
\begin{center}
    \includegraphics[width=200px]{../../pics/css_1_03_symm_krypto_RMU_page_13_1.png}
  \includegraphics[width=200px]{../../pics/css_1_03_symm_krypto_RMU_page_14_1.png}
\end{center}
\end{document}